\documentclass[]{article}
\usepackage{lmodern}
\usepackage{amssymb,amsmath}
\usepackage{ifxetex,ifluatex}
\usepackage{fixltx2e} % provides \textsubscript
\ifnum 0\ifxetex 1\fi\ifluatex 1\fi=0 % if pdftex
  \usepackage[T1]{fontenc}
  \usepackage[utf8]{inputenc}
\else % if luatex or xelatex
  \ifxetex
    \usepackage{mathspec}
  \else
    \usepackage{fontspec}
  \fi
  \defaultfontfeatures{Ligatures=TeX,Scale=MatchLowercase}
\fi
% use upquote if available, for straight quotes in verbatim environments
\IfFileExists{upquote.sty}{\usepackage{upquote}}{}
% use microtype if available
\IfFileExists{microtype.sty}{%
\usepackage{microtype}
\UseMicrotypeSet[protrusion]{basicmath} % disable protrusion for tt fonts
}{}
\usepackage[margin=1in]{geometry}
\usepackage{hyperref}
\hypersetup{unicode=true,
            pdftitle={Combining global tree cover loss data with historical national forest-cover maps to look at six decades of deforestation and forest fragmentation in Madagascar},
            pdfborder={0 0 0},
            breaklinks=true}
\urlstyle{same}  % don't use monospace font for urls
\usepackage{longtable,booktabs}
\usepackage{graphicx,grffile}
\makeatletter
\def\maxwidth{\ifdim\Gin@nat@width>\linewidth\linewidth\else\Gin@nat@width\fi}
\def\maxheight{\ifdim\Gin@nat@height>\textheight\textheight\else\Gin@nat@height\fi}
\makeatother
% Scale images if necessary, so that they will not overflow the page
% margins by default, and it is still possible to overwrite the defaults
% using explicit options in \includegraphics[width, height, ...]{}
\setkeys{Gin}{width=\maxwidth,height=\maxheight,keepaspectratio}
\IfFileExists{parskip.sty}{%
\usepackage{parskip}
}{% else
\setlength{\parindent}{0pt}
\setlength{\parskip}{6pt plus 2pt minus 1pt}
}
\setlength{\emergencystretch}{3em}  % prevent overfull lines
\providecommand{\tightlist}{%
  \setlength{\itemsep}{0pt}\setlength{\parskip}{0pt}}
\setcounter{secnumdepth}{5}
% Redefines (sub)paragraphs to behave more like sections
\ifx\paragraph\undefined\else
\let\oldparagraph\paragraph
\renewcommand{\paragraph}[1]{\oldparagraph{#1}\mbox{}}
\fi
\ifx\subparagraph\undefined\else
\let\oldsubparagraph\subparagraph
\renewcommand{\subparagraph}[1]{\oldsubparagraph{#1}\mbox{}}
\fi

%%% Use protect on footnotes to avoid problems with footnotes in titles
\let\rmarkdownfootnote\footnote%
\def\footnote{\protect\rmarkdownfootnote}

%%% Change title format to be more compact
\usepackage{titling}

% Create subtitle command for use in maketitle
\newcommand{\subtitle}[1]{
  \posttitle{
    \begin{center}\large#1\end{center}
    }
}

\setlength{\droptitle}{-2em}
  \title{Combining global tree cover loss data with historical national
forest-cover maps to look at six decades of deforestation and forest
fragmentation in Madagascar}
  \pretitle{\vspace{\droptitle}\centering\huge}
  \posttitle{\par}
  \author{}
  \preauthor{}\postauthor{}
  \date{}
  \predate{}\postdate{}

\usepackage{booktabs}
\usepackage{longtable}
\usepackage{array}
\usepackage{multirow}
\usepackage[table]{xcolor}
\usepackage{wrapfig}
\usepackage{float}
\usepackage{colortbl}
\usepackage{pdflscape}
\usepackage{tabu}
\usepackage{threeparttable}
\usepackage[normalem]{ulem}

\usepackage{amsthm}
\newtheorem{theorem}{Theorem}[section]
\newtheorem{lemma}{Lemma}[section]
\theoremstyle{definition}
\newtheorem{definition}{Definition}[section]
\newtheorem{corollary}{Corollary}[section]
\newtheorem{proposition}{Proposition}[section]
\theoremstyle{definition}
\newtheorem{example}{Example}[section]
\theoremstyle{definition}
\newtheorem{exercise}{Exercise}[section]
\theoremstyle{remark}
\newtheorem*{remark}{Remark}
\newtheorem*{solution}{Solution}
\begin{document}
\maketitle

\textbf{Ghislain Vieilledent\(^{1,2,\star}\), Clovis Grinand\(^{3}\),
Fety A. Rakotomalala\(^{3}\), Rija Ranaivosoa\(^{4}\), Jean-Roger
Rakotoarijaona\(^{4}\), Thomas F. Allnutt\(^{5,6}\), and Frédéric
Achard\(^{1}\)}

\begin{enumerate}
\def\labelenumi{\arabic{enumi}.}
\tightlist
\item
  Joint Research Center of the European Commission, Bio-economy Unit
  (JRC.D.1), I-21027 Ispra (VA), ITALY
\item
  Cirad, UPR Forêts et Sociétés, F-34398 Montpellier, FRANCE
\item
  ETC Terra, F-75016 Paris, FRANCE
\item
  Office National pour l'Environnement, 101 Antananarivo, MADAGASCAR
\item
  Wildlife Conservation Society, 101 Antananarivo, MADAGASCAR
\item
  GreenInfo Network, Oakland, California, USA
\end{enumerate}

(*) /Email:
\href{mailto:ghislain.vieilledent@cirad.fr}{\nolinkurl{ghislain.vieilledent@cirad.fr}},
/Phone: +39.329.457.2273

\emph{Running headline:} Six decades of deforestation in Madagascar

\hypertarget{highlights}{%
\section{Highlights}\label{highlights}}

\begin{itemize}
\tightlist
\item
  We produced new 30m-resolution annual forest-cover maps for Madagascar
  for the period 2000-2014.
\item
  Madagascar has lost 44\% of its natural forest-cover over the period
  1953-2014.
\item
  Half of the tropical forest in Madagascar is now located at less than
  100m from forest edge.
\item
  Annual deforestation rate has increased in Madagascar since 2005 to
  reach 1.08\%/yr.
\item
  Conservation and development efforts must be intensified to save
  Madagascar forest and biodiversity.
\end{itemize}

\hypertarget{summary}{%
\section{Summary}\label{summary}}

The island of Madagascar has a unique biodiversity, mainly located in
the tropical forests of the island. This biodiversity is highly
threatened by anthropogenic deforestation. Existing historical forest
maps at national level are scattered and have substantial gaps which
prevent an exhaustive assessment of long-term deforestation trends in
Madagascar. In this study, we combine historical national forest cover
maps (covering the period 1953-2000) with a recent global annual tree
cover loss dataset (2001-2014) to look at six decades of deforestation
and forest fragmentation in Madagascar (from 1953 to 2014). We produced
new forest cover maps at 30~m resolution for the year 1990 and annually
from 2000 to 2014 over the full territory of Madagascar. We estimated
that Madagascar has lost 44\% of its natural forest cover over the
period 1953-2014 (including 37\% over the period 1973-2014). Natural
forests cover 8.9 Mha in 2014 (15\% of the national territory) and
include 4.4 Mha (50\%) of moist forests, 2.6 Mha (29\%) of dry forests,
1.7 Mha of spiny forests (19\%) and 177,000 ha (2\%) of mangroves. Since
2005, the annual deforestation rate has progressively increased in
Madagascar to reach 99,000 ha/yr during 2010-2014 (corresponding to a
rate of 1.1\%/yr). This increase is probably due to rapid population
growth (close to 3\%/yr) and to poor law enforcement in the country.
Around half of the forest (46\%) is now located at less than 100~m from
the forest edge. Accurate forest cover change maps can be used to assess
the effectiveness of past and current conservation programs and
implement new strategies for the future. In particular, forest maps and
estimates can be used in the REDD+ framework which aims at ``Reducing
Emissions from Deforestation and Forest Degradation'' and for optimizing
the current protected area network.

\textbf{Keywords}: biodiversity, climate-change, deforestation,
forest-fragmentation, Madagascar, tropical forest.

\hypertarget{introduction}{%
\section{Introduction}\label{introduction}}

Separated from the African continent and the Indian plate about 165 and
88 million years ago respectively (Ali \& Aitchison
\protect\hyperlink{ref-Ali2008}{2008}), the flora and fauna of
Madagascar followed its own evolutionary path. Isolation combined with a
high number of micro-habitats (Pearson \& Raxworthy
\protect\hyperlink{ref-Pearson2009}{2009}) has led to Madagascar's
exceptional biodiversity both in term of number of species and endemism
in many taxonomic groups (Goodman \& Benstead
\protect\hyperlink{ref-Goodman2005}{2005}; Crottini \emph{et al.}
\protect\hyperlink{ref-Crottini2012}{2012}). Most of the biodiversity in
Madagascar is concentrated in the tropical forests of the island which
can be divided into four types: the moist forest in the East, the dry
forest in the West, the spiny forest in the South and the mangroves on
the West coast (Vieilledent \emph{et al.}
\protect\hyperlink{ref-Vieilledent2016}{2016}). This unparalleled
biodiversity is severely threatened by deforestation (Harper \emph{et
al.} \protect\hyperlink{ref-Harper2007}{2007}; Vieilledent, Grinand \&
Vaudry \protect\hyperlink{ref-Vieilledent2013}{2013}) associated with
human activities such as slash-and-burn agriculture and pasture (Scales
\protect\hyperlink{ref-Scales2011}{2011}). Tropical forests in
Madagascar also store a large amount of carbon (Vieilledent \emph{et
al.} \protect\hyperlink{ref-Vieilledent2016}{2016}) and high rates of
deforestation in Madagascar are responsible for large CO\(_2\) emissions
in the atmosphere (Achard \emph{et al.}
\protect\hyperlink{ref-Achard2014}{2014}). Deforestation threatens
species survival by directly reducing their available habitat (Tidd,
Pinder \& Ferguson \protect\hyperlink{ref-Tidd2001}{2001}; Brooks
\emph{et al.} \protect\hyperlink{ref-Brooks2002}{2002}). Forest
fragmentation can also lead to species extinction by isolating
populations from each other and creating forest patches too small to
maintain viable populations (Saunders, Hobbs \& Margules
\protect\hyperlink{ref-Saunders1991}{1991}). Fragmentation also
increases forest edge where ecological conditions (such as air
temperature, light intensity and air moisture) can be dramatically
modified, with consequences on the abundance and distribution of species
(Murcia \protect\hyperlink{ref-Murcia1995}{1995}). Forest fragmentation
can also have substantial effects on forest carbon storage capacity, as
carbon stocks are much lower at the forest edge than under a closed
canopy (Brinck \emph{et al.} \protect\hyperlink{ref-Brinck2017}{2017}).
Moreover, forest carbon stocks vary spatially due to climate or soil
factors (Saatchi \emph{et al.}
\protect\hyperlink{ref-Saatchi2011}{2011}; Vieilledent \emph{et al.}
\protect\hyperlink{ref-Vieilledent2016}{2016}). As a consequence,
accurate and spatially explicit maps of forest cover and forest cover
change are necessary to monitor biodiversity loss and carbon emissions
from deforestation and forest fragmentation, assess the efficiency of
present conservation strategies (Eklund \emph{et al.}
\protect\hyperlink{ref-Eklund2016}{2016}), and implement new strategies
for the future (Vieilledent, Grinand \& Vaudry
\protect\hyperlink{ref-Vieilledent2013}{2013}; Vieilledent \emph{et al.}
\protect\hyperlink{ref-Vieilledent2016}{2016}). Simple time-series of
forest cover estimates, such as those provided by the FAO Forest
Resource Assessment report (Keenan \emph{et al.}
\protect\hyperlink{ref-Keenan2015}{2015}) are not sufficient.

Unfortunately, accurate and exhaustive forest cover maps are not
available for Madagascar after year 2000. Harper \emph{et al.}
(\protect\hyperlink{ref-Harper2007}{2007}) produced maps of forest cover
and forest cover changes over Madagascar for the years \emph{c.}~1953,
\emph{c.}~1973, 1990 and 2000. The \emph{c.}~1953 forest map was derived
from the visual interpretation of aerial photography at coarse scale
(1/1,000,000). Forest maps for the years \emph{c.}~1973, 1990, and 2000
were obtained from supervised classification of Landsat satellite images
at 60~m resolution (for the year 1973) or 30~m resolution (for years
1990 and 2000) and can be used to derive more accurate estimates of
forest cover (89.5\% accuracy reported for the forest/non-forest map of
year 2000). Nonetheless, maps provided by Harper \emph{et al.}
(\protect\hyperlink{ref-Harper2007}{2007}) are not exhaustive (due to
the presence of clouds in the satellite imagery), e.g.~11 244~km2 are
mapped as unknown cover type for the year 2000. Using a similar
supervised classification approach as in Harper \emph{et al.}
(\protect\hyperlink{ref-Harper2007}{2007}), more recent maps have been
produced for the periods 2000-2005-2010 by national institutions, with
the technical support of international environmental NGOs (MEFT, USAID,
and CI \protect\hyperlink{ref-MEFT2009}{2009}; ONE, DGF, FTM, MNP, and
CI \protect\hyperlink{ref-ONE2013}{2013}). Another set of recent forest
cover maps using an advanced statistical tool for classification, the
Random Forest classifier (Grinand \emph{et al.}
\protect\hyperlink{ref-Grinand2013}{2013}; Rakotomala \emph{et al.}
\protect\hyperlink{ref-Rakotomalala2015}{2015}), was produced for the
periods 2005-2010-2013 (ONE, DGF, MNP, WCS, and Etc Terra
\protect\hyperlink{ref-ONE2015}{2015}). However, these maps are either
too old to give recent estimates of deforestation (MEFT, USAID, and CI
\protect\hyperlink{ref-MEFT2009}{2009}; ONE, DGF, FTM, MNP, and CI
\protect\hyperlink{ref-ONE2013}{2013}), include large areas of missing
information due to images with high percentage of cloud cover (ONE, DGF,
FTM, MNP, and CI \protect\hyperlink{ref-ONE2013}{2013}), or show large
mis-classification in specific areas, especially in the dry and spiny
forest domain for which the spectral answer has a strong seasonal
behavior due to the deciduousness of such forests (overall accuracy is
lower than 0.8 for the dry and spiny forests for the maps produced by
ONE, DGF, MNP, WCS, and Etc Terra
(\protect\hyperlink{ref-ONE2015}{2015})). Moreover, the production of
such forest maps from a supervised classification approach requires
significant resources, especially regarding the image selection step
(required to minimize cloud cover) and the training step (visual
interpretation of a large number of polygons needed to train the
classification algorithm) (Rakotomala \emph{et al.}
\protect\hyperlink{ref-Rakotomalala2015}{2015}). Most of this work of
image selection and visual interpretation would need to be repeated to
produce new forest maps in the future using a similar approach.

Global forest or tree cover products have also been published recently
and can be tested at the national scale for Madagascar. Kim \emph{et
al.} (\protect\hyperlink{ref-Kim2014}{2014}) produced a global forest
cover change map from 1990 to 2000 (derived from Landsat imagery). This
product was updated to cover the period 1975-2005
(\url{http://glcf.umd.edu/data/landsatFCC/}) but forest cover maps after
2005 were not produced. Moreover, the approach used in Kim \emph{et al.}
(\protect\hyperlink{ref-Kim2014}{2014}) did not accurately map the
forests in the dry and spiny ecosystems of Madagascar (see Fig. 8 in Kim
\emph{et al.} (\protect\hyperlink{ref-Kim2014}{2014})). Hansen \emph{et
al.} (\protect\hyperlink{ref-Hansen2013}{2013}) mapped tree cover
percentage, annual tree cover loss and gain from 2000 to 2012 at global
scale at 30 m resolution. This product has since been updated and is now
available up to the year 2014 (Hansen \emph{et al.}
\protect\hyperlink{ref-Hansen2013}{2013}). To map forest cover from the
Hansen \emph{et al.} (\protect\hyperlink{ref-Hansen2013}{2013}) product,
a tree cover threshold must be selected (that defines forest cover).
Selecting such a threshold is not straightforward as the accuracy of the
global tree cover map strongly varies between forest types, and is
substantially lower for dry forests than for moist forests (Bastin
\emph{et al.} \protect\hyperlink{ref-Bastin2017}{2017}). Moreover, the
Hansen \emph{et al.} (\protect\hyperlink{ref-Hansen2013}{2013}) product
does not provide information on land-use. In particular the global tree
cover map does not separate tree plantations such as oil palm or
eucalyptus plantations from natural forests (Tropek \emph{et al.}
\protect\hyperlink{ref-Tropek2014}{2014}). Thus, the global tree cover
map from Hansen \emph{et al.} (\protect\hyperlink{ref-Hansen2013}{2013})
cannot be used alone to produce a map of forest cover (Tyukavina
\emph{et al.} \protect\hyperlink{ref-Tyukavina2017}{2017}).

In this study, we present a simple approach which combines the
historical forest maps from Harper \emph{et al.}
(\protect\hyperlink{ref-Harper2007}{2007}) and more recent global
products from Hansen \emph{et al.}
(\protect\hyperlink{ref-Hansen2013}{2013}) to derive annual wall-to-wall
forest cover change maps over the period 2000-2014 for Madagascar. We
use the forest cover map provided by Harper \emph{et al.}
(\protect\hyperlink{ref-Harper2007}{2007}) for the year 2000 (defining
the land-use) with the tree cover loss product provided by Hansen
\emph{et al.} (\protect\hyperlink{ref-Hansen2013}{2013}) that we apply
only inside forest areas identified by Harper \emph{et al.}
(\protect\hyperlink{ref-Harper2007}{2007}). Similar to the approach of
Harper \emph{et al.} (\protect\hyperlink{ref-Harper2007}{2007}), we also
assess trends in deforestation rates and forest fragmentation from
\emph{c.}~1953 to 2014. We finally discuss the possibility to extend our
approach to other tropical countries or repeat it in the future. We also
discuss how our results could help assess the effectiveness of current
conservation strategies, and implementation new conservation strategies
for the future in Madagascar.

\hypertarget{materials-and-methods}{%
\section{Materials and Methods}\label{materials-and-methods}}

\hypertarget{creation-of-new-forest-cover-maps-of-madagascar-from-1953-to-2014}{%
\subsection{Creation of new forest cover maps of Madagascar from 1953 to
2014}\label{creation-of-new-forest-cover-maps-of-madagascar-from-1953-to-2014}}

We produced annual forest/non-forest maps at 30~m resolution for the
full territory of Madagascar for the period 2000-2014 by combining the
forest map of year 2000 from Harper \emph{et al.}
(\protect\hyperlink{ref-Harper2007}{2007}), and the tree cover
percentage and annual tree cover loss maps over the period 2000-2014
from Hansen \emph{et al.} (\protect\hyperlink{ref-Hansen2013}{2013}).
The 2000 Harper's forest map includes 208,000 ha of unclassified areas
due to the presence of clouds on satellite images, mostly (88\%) within
the moist forest domain which covered 4.17 Mha in total in 2000. To
provide a label (forest or non-forest) to these unclassified pixels, we
used the 2000 tree cover percentage map of Hansen \emph{et al.}
(\protect\hyperlink{ref-Hansen2013}{2013}) by selecting a threshold of
75\% tree cover to define forest cover as recommended by other studies
for the moist domain (Achard \emph{et al.}
\protect\hyperlink{ref-Achard2014}{2014}; Aleman, Jarzyna \& Staver
\protect\hyperlink{ref-Aleman2017}{2017}). To do so, the Hansen's 2000
tree cover map was resampled on the same grid as the original Harper's
map at 30~m resolution using a bilinear interpolation. We thus obtained
a forest cover map for the year 2000 covering the full territory of
Madagascar. We then combined this forest cover map of the year 2000 with
the annual tree cover loss maps from 2001 to 2014 provided by Hansen
\emph{et al.} (\protect\hyperlink{ref-Hansen2013}{2013}) to create
annual forest cover maps from 2001 to 2014 at 30~m resolution. To do so,
Hansen's tree cover loss maps were resampled on the same grid as the
original Harper's map at 30~m resolution using a nearest-neighbor
interpolation. We also completed the Harper's forest map of year 1990 by
filling unclassified areas (due to the presence of clouds on satellite
images) using our forest cover map of year 2000. To do so, we assumed
that if forest was present in 2000, the pixel was also forested in 1990.
The remaining unclassified pixels were limited to a relatively small
total area of \emph{c.}~8,000 ha. We labeled these residual pixels as
non-forest, as for the year 2000. Similarly we completed the Harper's
forest map of year 1973 by filling unclassified areas using our forest
cover map of the year 1990 assuming that if forest was present in 1990,
it was also present in 1973. Contrary to the year 1990, the remaining
unclassified pixels for year 1973 corresponded to a significant total
area of 3.3 million ha. We also reprojected the forest cover map of year
1953 to a common projection in order to compare the forest cover area in
1953 with forest cover areas at the following dates. This map was
produced by scanning a paper map derived from aerial photos, and thus
could not be perfectly aligned with the other maps produced through
digital processing of satellite imagery (Harper \emph{et al.}
\protect\hyperlink{ref-Harper2007}{2007}). Finally for all forest cover
maps from 1973, the isolated single non-forest pixels (i.e.~fully
surrounded by forest pixels) were removed, assuming that single
non-forest pixels inside a forest patch were not corresponding to
deforestation (they might correspond to selective logging activities).
This allowed us to avoid counting very small scale events
(\textless{}0.1 ha, such as selective logging) as forest fragmentation.
All the resulting maps have been made permanently and freely available
on the Zenodo research data repository at
\url{https://doi.org/10.5281/zenodo.1145785}.

\hypertarget{computing-forest-cover-areas-and-deforestation-rates}{%
\subsection{Computing forest cover areas and deforestation
rates}\label{computing-forest-cover-areas-and-deforestation-rates}}

From these new forest cover maps, we calculated the total forest cover
area for seven available years (1953-1973-1990-2000-2005-2010-2014), and
the annual deforested area and annual deforestation rate for the
corresponding six time periods between 1953 and 2014. The annual
deforestation rates were calculated using Eq.~\eqref{eq:theta} (Puyravaud
\protect\hyperlink{ref-Puyravaud2003}{2003}; Vieilledent, Grinand \&
Vaudry \protect\hyperlink{ref-Vieilledent2013}{2013}):

\begin{equation}
  \label{eq:theta}
  \theta = 100 \times [1-(1-(F_{t_2}-F_{t_1})/F_{t_1})^{(1/(t_2-t_1))}]
\end{equation}

In Eq.~\eqref{eq:theta}, \(\theta\) is the annual deforestation rate (in
\%/yr), \(F_{t_2}\) and \(F_{t_1}\) are the forest cover free of clouds
at both dates \(t_2\) and \(t_1\), and \(t_2-t_1\) is the time-interval
(in years) between the two dates.

Because of the large unclassified area (3.3 million ha) in 1973, the
annual deforestation areas and rates for the two periods 1953-1973 and
1973-1990 are only partial estimates computed on the basis of the
available forest extent. Area and rate estimates are produced at the
national scale and for the four forest ecosystems present in Madagascar:
moist forest in the East, dry forest in the West, spiny forest in the
South, and mangroves on the Western coast (Fig.~\ref{fig:ecoregions}).
To define the forest domains, we used a map from the MEFT
(\emph{``Ministère de l'Environnement et des Forêts à Madagascar''})
with the boundaries of the four ecoregions in Madagascar. Ecoregions
were defined on the basis of climatic and vegetation criteria using the
climate classification by Cornet
(\protect\hyperlink{ref-Cornet1974}{1974}) and the vegetation
classification from the 1996 IEFN national forest inventory (Ministère
de l'Environnement \protect\hyperlink{ref-IEFN1996}{1996}). Because
mangrove forests are highly dynamic ecosystems that can expand or
contract on decadal scales depending on changes in environmental factors
(Armitage \emph{et al.} \protect\hyperlink{ref-Armitage2015}{2015}), a
fixed delimitation of the mangrove ecoregion on six decades might not be
fully appropriate. As a consequence, our estimates of the forest cover
and deforestation rates for mangroves in Madagascar must be considered
with this limitation.

\hypertarget{comparing-our-forest-cover-and-deforestation-rate-estimates-with-previous-studies}{%
\subsection{Comparing our forest cover and deforestation rate estimates
with previous
studies}\label{comparing-our-forest-cover-and-deforestation-rate-estimates-with-previous-studies}}

We compared our estimates of forest cover and deforestation rates with
estimates from the three existing studies at the national scale for
Madagascar: (i) Harper \emph{et al.}
(\protect\hyperlink{ref-Harper2007}{2007}), (ii) MEFT, USAID, and CI
(\protect\hyperlink{ref-MEFT2009}{2009}) and (iii) ONE, DGF, MNP, WCS,
and Etc Terra (\protect\hyperlink{ref-ONE2015}{2015}). Harper \emph{et
al.} (\protect\hyperlink{ref-Harper2007}{2007}) provides forest cover
and deforestation estimates for the periods c. 1953-c. 1973-1990-2000.
MEFT, USAID, and CI (\protect\hyperlink{ref-MEFT2009}{2009}) provides
estimates for the periods 1990-2000-2005 and ONE, DGF, MNP, WCS, and Etc
Terra (\protect\hyperlink{ref-ONE2015}{2015}) provides estimates for the
periods 2005-2010-2013. To compare our forest cover and deforestation
estimates over the same time periods, we consider an additional
time-period in our study (2010-2013) by creating an extra forest cover
map for the year 2013. We computed the Pearson's correlation coefficient
and the root mean square error (RMSE) between our forest cover estimates
and forest cover estimates from previous studies for all the dates and
forest types (including also the total forest cover estimates). For
previous studies, the computation of annual deforestation rates (in
\%/yr) is not always detailed and might slightly differ from one study
to another (see Puyravaud \protect\hyperlink{ref-Puyravaud2003}{2003}).
Harper \emph{et al.} (\protect\hyperlink{ref-Harper2007}{2007}) also
provide total deforested areas for the two periods 1973-1990 and
1990-2000. We converted these values into annual deforested area
estimates. When annual deforested areas were not reported (for 1953-1973
in Harper \emph{et al.} (\protect\hyperlink{ref-Harper2007}{2007}) and
in MEFT, USAID, and CI (\protect\hyperlink{ref-MEFT2009}{2009}) and ONE,
DGF, MNP, WCS, and Etc Terra (\protect\hyperlink{ref-ONE2015}{2015})),
we computed them from the forest cover estimates in each study. These
estimates cannot be corrected from the potential bias due to the
presence of residual clouds. Forest cover and deforestation rates were
then compared between all studies for the whole of Madagascar and the
four ecoregions. The same ecoregion boundaries as in our study were used
in ONE, DGF, MNP, WCS, and Etc Terra
(\protect\hyperlink{ref-ONE2015}{2015}) but this was not the case for
Harper \emph{et al.} (\protect\hyperlink{ref-Harper2007}{2007}) and
MEFT, USAID, and CI (\protect\hyperlink{ref-MEFT2009}{2009}), which can
explain a part of the differences between the estimates.

\hypertarget{fragmentation}{%
\subsection{Fragmentation}\label{fragmentation}}

We also conducted an analysis of changes in forest fragmentation for the
years 1953, 1973, 1990, 2000, 2005, 2010 and 2014 at 30~m resolution. We
used a moving window of \(51 \times 51\) pixels centered on each forest
pixel to compute the percentage of forest pixels in the neighborhood. We
used this percentage as an indication of the forest fragmentation. The
size of the moving windows was based on a compromise: a sufficiently
high number of cells (here 2601) had to be considered to be able to
compute a percentage and a reasonably low number of cells had to be
choosen to have a local estimate of the fragmentation. Computations were
done using the function \texttt{r.neighbors} of the GRASS GIS software
(Neteler \& Mitasova \protect\hyperlink{ref-Neteler2008}{2008}). Using
the density of forest in the neighborhood, we defined five forest
fragmentation classes: 0-20\% (highly fragmented), 21-40\%, 41-60\%,
61-80\% and 81-100\% (lowly fragmented). We reported the percentage of
forest falling in each fragmentation class for the six years and
analyzed the dynamics of fragmentation over the six decades.

We also computed the distance to forest edge for all forest pixels for
the years 1953, 1973, 1990, 2000, 2005, 2010 and 2014. For that, we used
the function \texttt{gdal\_proximity.py} of the GDAL library
(\url{http://www.gdal.org/}). We computed the mean and 90\% quantiles
(5\% and 95\%) of the distance to forest edge and looked at the
evolution of these values with time. Previous studies have shown that
forest micro-habitats were mainly altered within the first 100~m of the
forest edge (Murcia \protect\hyperlink{ref-Murcia1995}{1995}; Broadbent
\emph{et al.} \protect\hyperlink{ref-Broadbent2008}{2008}; Gibson
\emph{et al.} \protect\hyperlink{ref-Gibson2013}{2013}; Brinck \emph{et
al.} \protect\hyperlink{ref-Brinck2017}{2017}). Consequently, we also
estimated the percentage of forest within the first 100~m of the forest
edge for each year and looked at the evolution of this percentage over
the six decades.

\hypertarget{results}{%
\section{Results}\label{results}}

\hypertarget{forest-cover-change-and-deforestation-rates}{%
\subsection{Forest cover change and deforestation
rates}\label{forest-cover-change-and-deforestation-rates}}

Natural forests in Madagascar covered 16.0 Mha in 1953, about 27\% of
the national territory of 587,041 km2. In 2014, the forest cover dropped
to 8.9 Mha, corresponding to about 15\% of the national territory
(Fig.\textasciitilde{}\ref{fig:fcc} and Tab.~@ref(tab:forest\_cover)).
Madagascar has lost 44\% and 37\% of its natural forests between 1953
and 2014, and between 1973 and 2014 respectively
(Fig.\textasciitilde{}\ref{fig:fcc} and Tab.~@ref(tab:forest\_cover)).
In 2014 the remaining 8.9 Mha of natural forest were distributed as
follow: 4.4 Mha of moist forest (50\% of total forest cover), 2.6 Mha of
dry forest (29\%), 1.7 Mha of spiny forest (19\%) and 0.18 Mha (2\%) of
mangrove forest (Fig.\textasciitilde{}\ref{fig:ecoregion} and
Tab.~@ref(tab:comp\_forest)). Regarding the deforestation trend, we
observed a progressive decrease of the deforestation rate after 1990
from 205,000 ha/yr (1.6\%/yr) over the period 1973-1990 to 42,000 ha/yr
(0.4\%/yr) over the period 2000-2005 (Tab.~@ref(tab:forest\_cover)).
Then from 2005, the deforestation rate has progressively increased and
has more than doubled over the period 2010-2014 (99,000 ha/yr, 1.1\%/yr)
compared to 2000-2005 (Tab.~@ref(tab:forest\_cover)). The deforestation
trend, characterized by a progressive decrease of the deforestation rate
over the period 1990-2005 and a progressive increase of the
deforestation after 2005, is valid for all four forest types except the
spiny forest (Tab.~@ref(tab:comp\_defor)). For the spiny forest, the
deforestation rate during the period 2010-2013 was lower than on the
period 2005-2010 (Tab.~@ref(tab:comp\_defor)).

\hypertarget{comparison-with-previous-forest-cover-change-studies-in-madagascar}{%
\subsection{Comparison with previous forest cover change studies in
Madagascar}\label{comparison-with-previous-forest-cover-change-studies-in-madagascar}}

Forest cover maps provided by previous studies over Madagascar were not
exhaustive (unclassified areas) due to the presence of clouds on
satellite images used to produce such maps. In Harper \emph{et al.}
(\protect\hyperlink{ref-Harper2007}{2007}), the maps of years 1990 and
2000 include 0.5 and 1.12 Mha of unknown cover type respectively.
Proportions of unclassified areas are not reported in the two other
existing studies at the national level by MEFT, USAID, and CI
(\protect\hyperlink{ref-MEFT2009}{2009}) and ONE, DGF, MNP, WCS, and Etc
Terra (\protect\hyperlink{ref-ONE2015}{2015}). With our approach, we
produced wall-to-wall forest cover change maps from 1990 to 2014 for the
full territory of Madagascar (Tab.~@ref(tab:forest\_cover)). This
allowed us to produce more robust estimates of forest cover and
deforestation rates over this period. Our forest cover estimates over
the period 1953-2013 (considering forest cover estimates at national
level and by ecoregions for all the available dates) were well
correlated (Pearson's correlation coefficient = 0.99) to estimates from
the three previous studies (Tab.~@ref(tab:comp\_forest)) with a RMSE of
300,000 ha (6\% of the mean forest cover of 4.8 Mha when considering all
dates and forest types together). These small differences can be partly
attributed to differences in ecoregion boundaries. Despite significant
differences in deforestation estimates (Tab.~@ref(tab:comp\_defor)), a
similar deforestation trend was observed across studies with a decrease
of deforestation rates over the period 1990-2005, followed by a
progressive increase of the deforestation after 2005.

\hypertarget{evolution-of-forest-fragmentation-with-time}{%
\subsection{Evolution of forest fragmentation with
time}\label{evolution-of-forest-fragmentation-with-time}}

In parallel to the dynamics of deforestation, forest fragmentation has
progressively increased since 1953 in Madagascar. We observed a
continuous decrease of the mean distance to forest edge from 1953 to
2014 in Madagascar. The mean distance to forest edge has decreased to
\emph{c.}~300~m in 2014 while it was previously \emph{c.}~1.5~km in 1973
(Fig.\textasciitilde{}\ref{fig:dist_edge}). Moreover, a large proportion
(73\%) of the forest was located at a distance greater than 100~m in
1973, while almost half of the forest (46\%) is now at a distance lower
than 100\textasciitilde{}m from forest edge in 2014
(Fig.\textasciitilde{}\ref{fig:dist_edge}). The percentage of lowly
fragmented forest in Madagascar has continuously decreased since 1953.
The percentage of forest belonging to the lowly fragmented category has
fallen from 57\% in 1973 to 44\% in 2014. In 2014, 22\% of the forest
belonged to the two highest fragmented forest classes (less than 40\% of
forest cover in the neighborhood) while 15\% of the forest belonged to
these two classes in 1973 (Tab.~\ref{tab:frag}).

\hypertarget{discussion}{%
\section{Discussion}\label{discussion}}

\hypertarget{advantages-of-combining-recent-global-annual-tree-cover-loss-data-with-historical-national-forest-cover-maps}{%
\subsection{Advantages of combining recent global annual tree cover loss
data with historical national forest cover
maps}\label{advantages-of-combining-recent-global-annual-tree-cover-loss-data-with-historical-national-forest-cover-maps}}

In this study, we combined recent (2001-2014) global annual tree cover
loss data (Hansen \emph{et al.}
\protect\hyperlink{ref-Hansen2013}{2013}) with historical (1953-2000)
national forest cover maps (Harper \emph{et al.}
\protect\hyperlink{ref-Harper2007}{2007}) to look at six decades
(1953-2014) of deforestation and forest fragmentation in Madagascar. We
produced annual forest cover maps at 30~m resolution covering Madagascar
for the period 2000 to 2014. Our study extends the forest cover
monitoring on a six decades period (from 1953 to 2014) while harmonizing
the data from previous studies (Harper \emph{et al.}
\protect\hyperlink{ref-Harper2007}{2007}; MEFT, USAID, and CI
\protect\hyperlink{ref-MEFT2009}{2009}; ONE, DGF, MNP, WCS, and Etc
Terra \protect\hyperlink{ref-ONE2015}{2015}). We propose a generic
approach to solve the problem of forest definition which is needed to
transform the 2000 global tree cover dataset from Hansen \emph{et al.}
(\protect\hyperlink{ref-Hansen2013}{2013}) into a forest/non-forest map
(Tropek \emph{et al.} \protect\hyperlink{ref-Tropek2014}{2014}). We
propose to use a historical national forest cover map, based on a
national forest definition, as a forest cover mask. This approach could
be easily extended to other regions or countries for which an accurate
forest cover map is available at any date within the period 2000-2014,
but preferably at the beginning of the period to profit from the full
record and derive long-term estimates of deforestation. Moreover, this
approach can be repeated in the future if and when the global tree cover
product is updated. We have made the \texttt{R/GRASS} code used for this
study freely available in a GitHub repository (see Data availability
statement) to facilitate application to other study areas or repeat the
analysis in the future for Madagascar.

The accuracy of the derived forest cover change maps depends directly on
the accuracies of the historical forest cover maps and the tree cover
loss dataset. Using visual-interpretation of aerial images in 342 areas
distributed among all forest types, Harper \emph{et al.}
(\protect\hyperlink{ref-Harper2007}{2007}) estimated an overall 89.5\%
accuracy in identifying forest/non-forest classes for the year 2000. The
accuracy assessment of the tree cover loss dataset for the tropical
biome reported 13\% of false positives and 16.9\% of false negatives
(see Tab. S5 in Hansen \emph{et al.}
(\protect\hyperlink{ref-Hansen2013}{2013})). These numbers rise at
20.7\% and 20.6\% respectively for the subtropical biome. In the
subtropical biome, the lower density tree cover canopy makes it
difficult to detect change from tree cover to bare ground. For six
countries in Central Africa, with a majority of moist dense forest,
Verhegghen \emph{et al.} (\protect\hyperlink{ref-Verhegghen2016}{2016})
have compared deforestation estimates derived from the global tree cover
loss dataset (Hansen \emph{et al.}
\protect\hyperlink{ref-Hansen2013}{2013}) with results derived from
semi-automated supervised classification of Landsat satellite images
(Achard \emph{et al.} \protect\hyperlink{ref-Achard2014}{2014}) and they
found a good agreement between the two sets of estimates. Therefore, our
forest cover change maps after 2000 might be more accurate for the dense
moist forest than for the dry and spiny forest. In another study
assessing the accuracy of the tree cover loss product accross the
tropics (Tyukavina \emph{et al.}
\protect\hyperlink{ref-Tyukavina2015}{2015}), authors reported 4\% of
false positives and 48\% of false negatives in Sub-Saharian Africa. They
showed that 85\% of missing loss occured on the edges of other loss
patches. This means that tree cover loss might be underestimated in
Sub-Saharian Africa, probably due to the prevalence of small-scale
disturbance which is hard to map at 30~m, but that areas of large-scale
deforestation are well identified and spatial variability of the
deforestation is well represented. A proper accuracy assessment of our
forest cover change maps should be performed to better estimate the
uncertainty surrounding our forest cover change estimates in Madagascar
from year 2000 (Olofsson \emph{et al.}
\protect\hyperlink{ref-Olofsson2013}{2013},
\protect\hyperlink{ref-Olofsson2014}{2014}). Despite this limitation, we
have shown that the deforestation trend we observed for Madagascar, with
a doubling deforestation on the period 2010-2014 compared to 2000-2005,
was consistent with the other studies at the national scale (MEFT,
USAID, and CI \protect\hyperlink{ref-MEFT2009}{2009}; ONE, DGF, MNP,
WCS, and Etc Terra \protect\hyperlink{ref-ONE2015}{2015}).

Consistent with Harper \emph{et al.}
(\protect\hyperlink{ref-Harper2007}{2007}), we did not consider
potential forest regrowth in Madagascar (although Hansen \emph{et al.}
(\protect\hyperlink{ref-Hansen2013}{2013}) provided a tree cover gains
layer for the period 2001-2013) for several reasons. First, the tree
gain layer of Hansen \emph{et al.}
(\protect\hyperlink{ref-Hansen2013}{2013}) includes and catches more
easily tree plantations than natural forest regrowth (Tropek \emph{et
al.} \protect\hyperlink{ref-Tropek2014}{2014}). Second, there is little
evidence of natural forest regeneration in Madagascar (Grouzis \emph{et
al.} \protect\hyperlink{ref-Grouzis2001}{2001}; Harper \emph{et al.}
\protect\hyperlink{ref-Harper2007}{2007}). This can be explained by
several ecological processes following burning practice such as soil
erosion (Grinand \emph{et al.}
\protect\hyperlink{ref-Grinand2017}{2017}) and reduced seed bank due to
fire and soil loss (Grouzis \emph{et al.}
\protect\hyperlink{ref-Grouzis2001}{2001}). Moreover, in areas where
forest regeneration is ecologically possible, young forest regrowth are
more easily re-burnt for agriculture and pasture. Third, young secondary
forests provide more limited ecosystem services compared to old-growth
natural forests in terms of biodiversity and carbon storage.

\hypertarget{natural-forest-cover-change-in-madagascar-from-1953-to-2014}{%
\subsection{Natural forest cover change in Madagascar from 1953 to
2014}\label{natural-forest-cover-change-in-madagascar-from-1953-to-2014}}

We estimated that natural forest in Madagascar covers 8.9 Mha in 2014
(corresponding to 15\% of the country) and that Madagascar has lost 44\%
of its natural forest since 1953 (37\% since 1973). There is ongoing
scientific debate about the extent of the ``original'' forest cover in
Madagascar, and the extent to which humans have altered the natural
forest landscapes since their large-scale settlement around 800 CE (Cox
\emph{et al.} \protect\hyperlink{ref-Cox2012}{2012}; Burns \emph{et al.}
\protect\hyperlink{ref-Burns2016}{2016}). Early French naturalists
stated that the full island was originally covered by forest (Perrier de
La Bâthie \protect\hyperlink{ref-Perrier1921}{1921}; Humbert
\protect\hyperlink{ref-Humbert1927}{1927}), leading to the common
statement that 90\% of the natural forests have disappeared since the
arrival of humans on the island (Kull
\protect\hyperlink{ref-Kull2000}{2000}). More recent studies
counter-balanced that point of view saying that extensive areas of
grassland existed in Madagascar long before human arrival and were
determined by climate, natural grazing and other natural factors
(Virah-Sawmy \protect\hyperlink{ref-Virah-Sawmy2009}{2009}; Vorontsova
\emph{et al.} \protect\hyperlink{ref-Vorontsova2017}{2016}). Other
authors have questioned the entire narrative of extensive alteration of
the landscape by early human activity which, through legislation, has
severe consequences on local people (Kull
\protect\hyperlink{ref-Kull2000}{2000}; Klein
\protect\hyperlink{ref-Klein2002}{2002}). Whatever the original
proportion of natural forests and grasslands in Madagascar, our results
demonstrate that human activities since the 1950s have profoundly
impacted the natural tropical forests and that conservation and
development programs in Madagascar have failed to stop deforestation in
the recent years. Deforestation has strong consequences on biodiversity
and carbon emissions in Madagascar. Around 90\% of Madagascar's species
are forest dependent (Goodman \& Benstead
\protect\hyperlink{ref-Goodman2005}{2005}; Allnutt \emph{et al.}
\protect\hyperlink{ref-Allnutt2008}{2008}) and Allnutt \emph{et al.}
(\protect\hyperlink{ref-Allnutt2008}{2008}) estimated that deforestation
between 1953 and 2000 led to an extinction of 9\% of the species. The
additional deforestation we observed over the period 2000-2014 (around
1Mha of natural forest) worsen this result. Regarding carbon emissions,
using the 2010 aboveground forest carbon map by Vieilledent \emph{et
al.} (\protect\hyperlink{ref-Vieilledent2016}{2016}), we estimated that
deforestation on the period 2010-2014 has led to 40.2 Mt C of carbon
emissions in the atmosphere (10 Mt C /yr) and that the remaining
aboveground forest carbon stock in 2014 is 832.8 Mt C. Associated to
deforestation, we showed that the remaining forests of Madagascar are
highly fragmented with 46\% of the forest being at less than 100~m of
the forest edge. Small forest fragments do not allow to maintain viable
populations and ``edge effects'' at forest/non-forest interfaces have
impacts on both carbon emissions (Brinck \emph{et al.}
\protect\hyperlink{ref-Brinck2017}{2017}) and biodiversity loss (Murcia
\protect\hyperlink{ref-Murcia1995}{1995}; Gibson \emph{et al.}
\protect\hyperlink{ref-Gibson2013}{2013}).

\hypertarget{deforestation-trend-and-impacts-on-conservation-and-development-policies}{%
\subsection{Deforestation trend and impacts on conservation and
development
policies}\label{deforestation-trend-and-impacts-on-conservation-and-development-policies}}

In our study, we have shown that the progressive decrease of the
deforestation rate on the period 1990-2005 was followed by a continuous
increase in the deforestation rate on the period 2005-2014. In
particular, we showed that deforestation rate has more than doubled on
the period 2010-2014 compared to 2000-2005. Our results are confirmed by
previous studies (Harper \emph{et al.}
\protect\hyperlink{ref-Harper2007}{2007}; MEFT, USAID, and CI
\protect\hyperlink{ref-MEFT2009}{2009}; ONE, DGF, MNP, WCS, and Etc
Terra \protect\hyperlink{ref-ONE2015}{2015}) despite differences in the
methodologies regarding (i) forest definition (associated to independent
visual interpretations of observation polygons to train the classifier),
(ii) classification algorithms, (iii) deforestation rate computation
method, and (iv) correction for the presence of clouds. Our
deforestation rate estimates from 1990 to 2014 have been computed from
wall-to-wall maps at 30~m resolution and can be considered more accurate
in comparison with estimates from these previous studies. Our forest
cover and deforestation rate estimates can be used as source of
information for the next FAO Forest Resources Assessment (Keenan
\emph{et al.} \protect\hyperlink{ref-Keenan2015}{2015}). Current rates
of deforestation can also be used to build reference scenarios for
deforestation in Madagascar and contribute to the implementation of
deforestation mitigation activities in the framework of REDD+ (Olander
\emph{et al.} \protect\hyperlink{ref-Olander2008}{2008}).

The increase of deforestation rates after 2005 can be explained by
population growth and political instability in the country. Nearly 90\%
of Madagascar's population relies on biomass for their daily energy
needs (Minten, Sander \& Stifel
\protect\hyperlink{ref-Minten2013}{2013}) and the link between
population size and deforestation has previously been demonstrated in
Madagascar (Gorenflo \emph{et al.}
\protect\hyperlink{ref-Gorenflo2011}{2011}; Vieilledent, Grinand \&
Vaudry \protect\hyperlink{ref-Vieilledent2013}{2013}). With a mean
demographic growth rate of about 2.8\%/yr and a population which has
increased from 16 to 24 million people on the period 2000-2015 (United
Nations \protect\hyperlink{ref-UN2015}{2015}), the increasing demand in
wood-fuel and space for agriculture is likely to explain the increase in
deforestation rates. The political crisis of 2009 (Ploch \& Cook
\protect\hyperlink{ref-Ploch2012}{2012}), followed by several years of
political instability and weak governance could also explain the
increase in the deforestation rate observed on the period 2005-2014
(Smith \emph{et al.} \protect\hyperlink{ref-Smith2003}{2003}). These
results show that despite the conservation policy in Madagascar
(Freudenberger \protect\hyperlink{ref-Freudenberger2010}{2010}),
deforestation has dramatically increased at the national level since
2005. Results of this study, including recent spatially explicit forest
cover change maps and forest cover estimates, should help implement new
conservation strategies to save Madagascar natural tropical forests and
their unique biodiversity.

\hypertarget{authors-contribution}{%
\section{Author's contribution}\label{authors-contribution}}

All authors conceived the ideas and designed methodology; GV analysed
the data and wrote the \texttt{R/GRASS} script; GV drafted the
manuscript. All authors contributed critically to the drafts and gave
final approval for publication.

\hypertarget{acknowledgements}{%
\section{Acknowledgements}\label{acknowledgements}}

The authors thank Jean-François Bastin for useful comments on a previous
version of the manuscript and Peter Vogt for useful advices on which
metric to use to estimate forest fragmentation. This study is part of
the Cirad's BioSceneMada project (\url{https://bioscenemada.cirad.fr})
and the Joint Research Center's ReCaREDD project
(\url{http://forobs.jrc.ec.europa.eu/recaredd}). The BioSceneMada
project is funded by FRB (Fondation pour la Recherche sur la
Biodiversité) and the FFEM (Fond Français pour l'Environnement Mondial)
under the project agreement AAP-SCEN-2013 I. The ReCaREDD project is
funded by the European Commission. The authors declare that there are no
conflicts of interest related to this article.

\hypertarget{data-accessibility}{%
\section{Data accessibility}\label{data-accessibility}}

All the data and the script used for this study have been made
permanently and publicly available on the Zenodo research data
repository so that the results are entirely reproducible:

\begin{itemize}
\tightlist
\item
  Input data: \url{https://doi.org/10.5281/zenodo.1118955}
\item
  Script: \url{https://doi.org/10.5281/zenodo.1118484}
\item
  Output data: \url{https://doi.org/10.5281/zenodo.1145785}
\end{itemize}

\hypertarget{tables}{%
\section{Tables}\label{tables}}

\textbackslash{}begin\{table\}

\textbackslash{}caption\{(\#tab:forest\_cover)Test\} \centering

\begin{tabular}[t]{lrrrr}
\toprule
Year & Forest (Kha) & Unmap (Kha) & Annual defor. (Kha/yr) & Rate (\%/yr)\\
\midrule
1953 & 15,968 & 0 & - & -\\
1973 & 14,243 & 3,317 & 86 & 0.6\\
1990 & 10,762 & 0 & 205 & 1.6\\
2000 & 9,879 & 0 & 88 & 0.8\\
2005 & 9,668 & 0 & 42 & 0.4\\
\addlinespace
2010 & 9,320 & 0 & 70 & 0.7\\
2014 & 8,925 & 0 & 99 & 1.1\\
\bottomrule
\end{tabular}

\textbackslash{}end\{table\}

\textbackslash{}begin\{table\}

\textbackslash{}caption\{(\#tab:comp\_forest)Test\} \centering

\begin{tabular}[t]{llrrrrrrrr}
\toprule
Forest type & Source & 1953 & 1973 & 1990 & 2000 & 2005 & 2010 & 2013 & 2014\\
\midrule
Total & Harper2007 & 15,996 & 14,173 & 10,606 & 8,982 & - & - & - & -\\
 & MEFT2009 & - & - & 10,650 & 9,678 & 9,413 & - & - & -\\
 & ONE2015 & - & - & - & - & 9,451 & 8,977 & 8,486 & -\\
 & this study & 15,968 & 14,243 & 10,762 & 9,879 & 9,668 & 9,320 & 9,051 & 8,925\\
Moist & Harper2007 & 8,766 & 6,876 & 5,234 & 4,167 & - & - & - & -\\
\addlinespace
 & MEFT2009 & - & - & 5,271 & 4,788 & 4,700 & - & - & -\\
 & ONE2015 & - & - & - & - & 4,556 & 4,457 & 4,345 & -\\
 & this study & 8,578 & 6,990 & 5,270 & 4,872 & 4,768 & 4,633 & 4,470 & 4,410\\
Dry & Harper2007 & 4,252 & 4,028 & 2,712 & 2,457 & - & - & - & -\\
 & MEFT2009 & - & - & 3,321 & 3,085 & 3,028 & - & - & -\\
\addlinespace
 & ONE2015 & - & - & - & - & 3,223 & 2,970 & 2,679 & -\\
 & this study & 4,762 & 4,435 & 3,225 & 2,941 & 2,881 & 2,735 & 2,642 & 2,596\\
Spiny & Harper2007 & 2,978 & 3,030 & 2,420 & 2,132 & - & - & - & -\\
 & MEFT2009 & - & - & 2,124 & 1,872 & 1,757 & - & - & -\\
 & ONE2015 & - & - & - & - & 1,682 & 1,559 & 1,467 & -\\
\addlinespace
 & this study & 2,463 & 2,583 & 2,055 & 1,858 & 1,811 & 1,744 & 1,731 & 1,713\\
Mangroves & Harper2007 & - & - & 240 & 226 & - & - & - & -\\
 & MEFT2009 & - & - & - & - & - & - & - & -\\
 & ONE2015 & - & - & - & - & 174 & 171 & 170 & -\\
 & this study & 143 & 200 & 181 & 178 & 177 & 177 & 177 & 177\\
\bottomrule
\end{tabular}

\textbackslash{}end\{table\}

\textbackslash{}begin\{table\}

\textbackslash{}caption\{(\#tab:comp\_defor)Test\} \centering

\begin{tabular}[t]{llrrrrrr}
\toprule
Forest type & Source & 1953-1973 & 1973-1990 & 1990-2000 & 2000-2005 & 2005-2010 & 2010-2013\\
\midrule
Total & Harper2007 & 91 (0.3) & 200 (1.7) & 81 (0.9) & - & - & -\\
 & MEFT2009 & - & - & 97 (0.8) & 53 (0.5) & - & -\\
 & ONE2015 & - & - & - & - & 95 (1.2) & 164 (1.5)\\
 & this study & 86 (0.6) & 205 (1.6) & 88 (0.9) & 42 (0.4) & 70 (0.7) & 90 (1.0)\\
Moist & Harper2007 & 94 (0.6) & 87 (1.7) & 32 (0.8) & - & - & -\\
\addlinespace
 & MEFT2009 & - & - & 48 (0.8) & 17 (0.4) & - & -\\
 & ONE2015 & - & - & - & - & 20 (0.5) & 37 (0.9)\\
 & this study & 79 (1.0) & 101 (1.6) & 40 (0.8) & 21 (0.4) & 27 (0.6) & 54 (1.2)\\
Dry & Harper2007 & 11 (0.2) & 77 (1.9) & 20 (0.7) & - & - & -\\
 & MEFT2009 & - & - & 24 (0.7) & 11 (0.4) & - & -\\
\addlinespace
 & ONE2015 & - & - & - & - & 51 (1.8) & 97 (2.3)\\
 & this study & 16 (0.4) & 71 (1.9) & 28 (0.9) & 12 (0.4) & 29 (1.0) & 31 (1.1)\\
Spiny & Harper2007 & -3 (-0.1) & 36 (1.2) & 28 (1.2) & - & - & -\\
 & MEFT2009 & - & - & 25 (1.2) & 23 (1.2) & - & -\\
 & ONE2015 & - & - & - & - & 25 (1.7) & 31 (1.7)\\
\addlinespace
 & this study & -6 (-0.2) & 31 (1.3) & 20 (1.0) & 9 (0.5) & 13 (0.7) & 4 (0.3)\\
Mangroves & Harper2007 & - & - & 1 (0.2) & - & - & -\\
 & MEFT2009 & - & - & - & - & - & -\\
 & ONE2015 & - & - & - & - & 0 (0.3) & 0 (0.2)\\
 & this study & -3 (-1.7) & 1 (0.6) & 0 (0.2) & 0 (0.0) & 0 (0.0) & 0 (0.0)\\
\bottomrule
\end{tabular}

\textbackslash{}end\{table\}

\begin{table}

\caption{\label{tab:frag}Test}
\centering
\begin{tabular}[t]{lrrrrrr}
\toprule
Year & Forest (Kha) & 0-20 & 21-40 & 41-60 & 61-80 & 81-100\\
\midrule
1953 & 15,968 & 0 & 1 & 8 & 12 & 78\\
1973 & 14,243 & 6 & 9 & 12 & 16 & 57\\
1990 & 10,762 & 7 & 10 & 13 & 17 & 53\\
2000 & 9,879 & 7 & 11 & 14 & 17 & 51\\
2005 & 9,673 & 8 & 11 & 14 & 18 & 49\\
\addlinespace
2010 & 9,320 & 8 & 12 & 15 & 18 & 47\\
2014 & 8,925 & 9 & 13 & 16 & 19 & 44\\
\bottomrule
\end{tabular}
\end{table}

\hypertarget{figures}{%
\section{Figures}\label{figures}}

\begin{figure}
\centering
\includegraphics{figs/ecoregion.png}
\caption{\textbf{Ecoregions and forest types in Madagascar.} Madagascar
can be divided into four climatic ecoregions with four forest types: the
moist forest in the East (green), the dry forest in the West (orange),
the spiny forest in the South (red), and the mangroves on the West coast
(blue). Ecoregions were defined following climatic (Cornet
\protect\hyperlink{ref-Cornet1974}{1974}) and vegetation (Ministère de
l'Environnement \protect\hyperlink{ref-IEFN1996}{1996}) criteria. The
dark grey areas represent the remaining natural forest cover for the
year 2014.}
\end{figure}

\begin{figure}
\centering
\includegraphics{figs/fig_fcc.png}
\caption{\textbf{Forest cover change on six decades from 1953 to 2014 in
Madagascar.} Forest cover changes from \emph{c.} 1973 to 2014 are shown
in the main figure, and forest cover in \emph{c.} 1953 is shown in the
bottom-right inset. Two zooms in the western dry (left part) and eastern
moist (right part) ecoregions present more detailed views of (from top
to bottom): forest cover in 1950s, forest cover change from \emph{c.}
1973 to 2014, forest fragmentation in 2014 and distance to forest edge
in 2014. Data on water bodies (blue) and water seasonality (light blue
for seasonal water to dark blue for permanent water) has been extracted
from Pekel \emph{et al.} (\protect\hyperlink{ref-Pekel2016}{2016}).}
\end{figure}

\begin{figure}
\centering
\includegraphics{figs/dist.png}
\caption{\textbf{Evolution of the distance to forest edge from 1953 to
2014 in Madagascar.} Black dots represent the mean distance to forest
edge for each year. Vertical dashed segments represent the 90\%
quantiles (5\% and 95\%) of the distance to forest edge. Horizontal
dashed grey line indicates a distance to forest edge of 100 m. Numbers
at the bottom of each vertical segments are the percentage of forest at
a distance to forest edge lower than 100 m for each year.}
\end{figure}

\hypertarget{references}{%
\section*{References}\label{references}}
\addcontentsline{toc}{section}{References}

\hypertarget{refs}{}
\leavevmode\hypertarget{ref-Achard2014}{}%
Achard, F., Beuchle, R., Mayaux, P., Stibig, H.-J., Bodart, C., Brink,
A., Carboni, S., Desclée, B., Donnay, F., Eva, H.D., Lupi, A., Raši, R.,
Seliger, R. \& Simonetti, D. (2014) Determination of tropical
deforestation rates and related carbon losses from 1990 to 2010.
\emph{Global Change Biology}, \textbf{20}, 2540--2554.

\leavevmode\hypertarget{ref-Aleman2017}{}%
Aleman, J.C., Jarzyna, M.A. \& Staver, A.C. (2017) Forest extent and
deforestation in tropical africa since 1900. \emph{Nature Ecology \&
Evolution}.

\leavevmode\hypertarget{ref-Ali2008}{}%
Ali, J.R. \& Aitchison, J.C. (2008) Gondwana to Asia: Plate tectonics,
paleogeography and the biological connectivity of the Indian
sub-continent from the Middle Jurassic through latest Eocene (166--35
Ma). \emph{Earth-Science Reviews}, \textbf{88}, 145--166.

\leavevmode\hypertarget{ref-Allnutt2008}{}%
Allnutt, T.F., Ferrier, S., Manion, G., Powell, G.V.N., Ricketts, T.H.,
Fisher, B.L., Harper, G.J., Irwin, M.E., Kremen, C., Labat, J.-N., Lees,
D.C., Pearce, T.A. \& Rakotondrainibe, F. (2008) A method for
quantifying biodiversity loss and its application to a 50-year record of
deforestation across Madagascar. \emph{Conservation Letters},
\textbf{1}, 173--181.

\leavevmode\hypertarget{ref-Armitage2015}{}%
Armitage, A.R., Highfield, W.E., Brody, S.D. \& Louchouarn, P. (2015)
The contribution of mangrove expansion to salt marsh loss on the Texas
Gulf Coast. \emph{PloS One}, \textbf{10}, e0125404.

\leavevmode\hypertarget{ref-Bastin2017}{}%
Bastin, J.-F., Berrahmouni, N., Grainger, A., Maniatis, D., Mollicone,
D., Moore, R., Patriarca, C., Picard, N., Sparrow, B., Abraham, E.M.,
Aloui, K., Atesoglu, A., Attore, F., Bassüllü, Ç., Bey, A., Garzuglia,
M., García-Montero, L.G., Groot, N., Guerin, G., Laestadius, L., Lowe,
A.J., Mamane, B., Marchi, G., Patterson, P., Rezende, M., Ricci, S.,
Salcedo, I., Diaz, A.S.-P., Stolle, F., Surappaeva, V. \& Castro, R.
(2017) The extent of forest in dryland biomes. \emph{Science},
\textbf{356}, 635--638.

\leavevmode\hypertarget{ref-Brinck2017}{}%
Brinck, K., Fischer, R., Groeneveld, J., Lehmann, S., Dantas De Paula,
M., Pütz, S., Sexton, J.O., Song, D. \& Huth, A. (2017) High resolution
analysis of tropical forest fragmentation and its impact on the global
carbon cycle. \emph{Nature Communications}, \textbf{8}, 14855.

\leavevmode\hypertarget{ref-Broadbent2008}{}%
Broadbent, E.N., Asner, G.P., Keller, M., Knapp, D.E., Oliveira, P.J. \&
Silva, J.N. (2008) Forest fragmentation and edge effects from
deforestation and selective logging in the brazilian amazon.
\emph{Biological Conservation}, \textbf{141}, 1745--1757.

\leavevmode\hypertarget{ref-Brooks2002}{}%
Brooks, T.M., Mittermeier, R.A., Mittermeier, C.G., Fonseca, G.A.B. da,
Rylands, A.B., Konstant, W.R., Flick, P., Pilgrim, J., Oldfield, S.,
Magin, G. \& Hilton-Taylor, C. (2002) Habitat loss and extinction in the
hotspots of biodiversity. \emph{Conservation Biology}, \textbf{16},
909--923.

\leavevmode\hypertarget{ref-Burns2016}{}%
Burns, S.J., Godfrey, L.R., Faina, P., McGee, D., Hardt, B.,
Ranivoharimanana, L. \& Randrianasy, J. (2016) Rapid human-induced
landscape transformation in Madagascar at the end of the first
millennium of the Common Era. \emph{Quaternary Science Reviews},
\textbf{134}, 92--99.

\leavevmode\hypertarget{ref-Cornet1974}{}%
Cornet, A. (1974) \emph{Essai de cartographie bioclimatique à
Madagascar}. Orstom.

\leavevmode\hypertarget{ref-Cox2012}{}%
Cox, M.P., Nelson, M.G., Tumonggor, M.K., Ricaut, F.-X. \& Sudoyo, H.
(2012) A small cohort of Island Southeast Asian women founded
Madagascar. \emph{Proceedings of the Royal Society B: Biological
Sciences}, \textbf{279}, 2761--2768.

\leavevmode\hypertarget{ref-Crottini2012}{}%
Crottini, A., Madsen, O., Poux, C., Strauß, A., Vieites, D.R. \& Vences,
M. (2012) Vertebrate time-tree elucidates the biogeographic pattern of a
major biotic change around the K--T boundary in Madagascar.
\emph{Proceedings of the National Academy of Sciences}, \textbf{109},
5358--5363.

\leavevmode\hypertarget{ref-Eklund2016}{}%
Eklund, J., Blanchet, F.G., Nyman, J., Rocha, R., Virtanen, T. \&
Cabeza, M. (2016) Contrasting spatial and temporal trends of protected
area effectiveness in mitigating deforestation in Madagascar.
\emph{Biological Conservation}, \textbf{203}, 290--297.

\leavevmode\hypertarget{ref-Freudenberger2010}{}%
Freudenberger, K. (2010) Paradise Lost? Lessons from 25 years of USAID
environment programs in Madagascar. \emph{International Resources Group,
Washington DC}.

\leavevmode\hypertarget{ref-Gibson2013}{}%
Gibson, L., Lynam, A.J., Bradshaw, C.J., He, F., Bickford, D.P.,
Woodruff, D.S., Bumrungsri, S. \& Laurance, W.F. (2013) Near-complete
extinction of native small mammal fauna 25 years after forest
fragmentation. \emph{Science}, \textbf{341}, 1508--1510.

\leavevmode\hypertarget{ref-Goodman2005}{}%
Goodman, S.M. \& Benstead, J.P. (2005) Updated estimates of biotic
diversity and endemism for Madagascar. \emph{Oryx}, \textbf{39}, 73--77.

\leavevmode\hypertarget{ref-Gorenflo2011}{}%
Gorenflo, L.J., Corson, C., Chomitz, K.M., Harper, G., Honzák, M. \&
Özler, B. (2011) \emph{Exploring the Association Between People and
Deforestation in Madagascar} (eds R.P. Cincotta \& L.J. Gorenflo).
Springer Berlin Heidelberg.

\leavevmode\hypertarget{ref-Grinand2017}{}%
Grinand, C., Le Maire, G., Vieilledent, G., Razakamanarivo, H.,
Razafimbelo, T. \& Bernoux, M. (2017) Estimating temporal changes in
soil carbon stocks at ecoregional scale in Madagascar using
remote-sensing. \emph{International Journal of Applied Earth Observation
and Geoinformation}, \textbf{54}, 1--14.

\leavevmode\hypertarget{ref-Grinand2013}{}%
Grinand, C., Rakotomalala, F., Gond, V., Vaudry, R., Bernoux, M. \&
Vieilledent, G. (2013) Estimating deforestation in tropical humid and
dry forests in Madagascar from 2000 to 2010 using multi-date Landsat
satellite images and the Random Forests classifier. \emph{Remote Sensing
of Environment}, \textbf{139}, 68--80.

\leavevmode\hypertarget{ref-Grouzis2001}{}%
Grouzis, M., Razanaka, S., Le Floc'h, E. \& Leprun, J.-C. (2001)
Évolution de la végétation et de quelques paramètres édaphiques au cours
de la phase post-culturale dans la région d'Analabo. \emph{Sociétés
paysannes, transitions agraires et dynamiques écologiques dans le
Sud-Ouest de Madagascar, Antananarivo, IRD/CNRE}, 327--337.

\leavevmode\hypertarget{ref-Hansen2013}{}%
Hansen, M.C., Potapov, P.V., Moore, R., Hancher, M., Turubanova, S.A.,
Tyukavina, A., Thau, D., Stehman, S.V., Goetz, S.J., Loveland, T.R.,
Kommareddy, A., Egorov, A., Chini, L., Justice, C.O. \& Townshend,
J.R.G. (2013) High-Resolution Global Maps of 21st-Century Forest Cover
Change. \emph{Science}, \textbf{342}, 850--853.

\leavevmode\hypertarget{ref-Harper2007}{}%
Harper, G.J., Steininger, M.K., Tucker, C.J., Juhn, D. \& Hawkins, F.
(2007) Fifty years of deforestation and forest fragmentation in
Madagascar. \emph{Environmental Conservation}, \textbf{34}, 325--333.

\leavevmode\hypertarget{ref-Humbert1927}{}%
Humbert, H. (1927) La destruction d'une flore insulaire par le feu.
Principaux aspects de la végétation à Madagascar. \emph{Mémoires de
l'Académie Malgache}, \textbf{5}, 1--80.

\leavevmode\hypertarget{ref-Keenan2015}{}%
Keenan, R.J., Reams, G.A., Achard, F., Freitas, J.V. de, Grainger, A. \&
Lindquist, E. (2015) Dynamics of global forest area: Results from the
FAO Global Forest Resources Assessment 2015. \emph{Forest Ecology and
Management}, \textbf{352}, 9--20.

\leavevmode\hypertarget{ref-Kim2014}{}%
Kim, D.-H., Sexton, J.O., Noojipady, P., Huang, C., Anand, A., Channan,
S., Feng, M. \& Townshend, J.R. (2014) Global, Landsat-based
forest-cover change from 1990 to 2000. \emph{Remote Sensing of
Environment}, \textbf{155}, 178--193.

\leavevmode\hypertarget{ref-Klein2002}{}%
Klein, J. (2002) Deforestation in the Madagascar Highlands --
Established `truth' and scientific uncertainty. \emph{GeoJournal},
\textbf{56}, 191--199.

\leavevmode\hypertarget{ref-Kull2000}{}%
Kull, C.A. (2000) Deforestation, erosion, and fire: degradation myths in
the environmental history of Madagascar. \emph{Environment and History},
\textbf{6}, 423--450.

\leavevmode\hypertarget{ref-MEFT2009}{}%
MEFT, USAID, and CI. (2009) \emph{Evolution de la couverture de forêts
naturelles à Madagascar, 1990-2000-2005}. Antananarivo.

\leavevmode\hypertarget{ref-IEFN1996}{}%
Ministère de l'Environnement. (1996) \emph{IEFN: Inventaire Ecologique
Forestier National}. Ministère de l'Environnement de Madagascar,
Direction des Eaux et Forêts, DFS Deutsch Forest Service GmbH,
Entreprise d'études de développement rural "Mamokatra", FTM.

\leavevmode\hypertarget{ref-Minten2013}{}%
Minten, B., Sander, K. \& Stifel, D. (2013) Forest management and
economic rents: Evidence from the charcoal trade in Madagascar.
\emph{Energy for Sustainable Development}, \textbf{17}, 106--115.

\leavevmode\hypertarget{ref-Murcia1995}{}%
Murcia, C. (1995) Edge effects in fragmented forests: Implications for
conservation. \emph{Trends in Ecology \& Evolution}, \textbf{10},
58--62.

\leavevmode\hypertarget{ref-Neteler2008}{}%
Neteler, M. \& Mitasova, H. (2008) \emph{Open source GIS: a GRASS GIS
approach}. Springer.

\leavevmode\hypertarget{ref-Olander2008}{}%
Olander, L.P., Gibbs, H.K., Steininger, M., Swenson, J.J. \& Murray,
B.C. (2008) Reference scenarios for deforestation and forest degradation
in support of REDD: a review of data and methods. \emph{Environmental
Research Letters}, \textbf{3}, 025011.

\leavevmode\hypertarget{ref-Olofsson2014}{}%
Olofsson, P., Foody, G.M., Herold, M., Stehman, S.V., Woodcock, C.E. \&
Wulder, M.A. (2014) Good practices for estimating area and assessing
accuracy of land change. \emph{Remote Sensing of Environment},
\textbf{148}, 42--57.

\leavevmode\hypertarget{ref-Olofsson2013}{}%
Olofsson, P., Foody, G.M., Stehman, S.V. \& Woodcock, C.E. (2013) Making
better use of accuracy data in land change studies: Estimating accuracy
and area and quantifying uncertainty using stratified estimation.
\emph{Remote Sensing of Environment}, \textbf{129}, 122--131.

\leavevmode\hypertarget{ref-ONE2013}{}%
ONE, DGF, FTM, MNP, and CI. (2013) \emph{Evolution de la couverture de
forêts naturelles à Madagascar 2005-2010}. Antananarivo.

\leavevmode\hypertarget{ref-ONE2015}{}%
ONE, DGF, MNP, WCS, and Etc Terra. (2015) \emph{Changement de la
couverture de forêts naturelles à Madagascar, 2005-2010-2013}.
Antananarivo.

\leavevmode\hypertarget{ref-Pearson2009}{}%
Pearson, R.G. \& Raxworthy, C.J. (2009) The evolution of local endemism
in Madagascar: watershed versus climatic gradient hypotheses evaluated
by null biogeographic models. \emph{Evolution}, \textbf{63}, 959--967.

\leavevmode\hypertarget{ref-Pekel2016}{}%
Pekel, J.-F., Cottam, A., Gorelick, N. \& Belward, A.S. (2016)
High-resolution mapping of global surface water and its long-term
changes. \emph{Nature}, \textbf{540}, 418--422.

\leavevmode\hypertarget{ref-Perrier1921}{}%
Perrier de La Bâthie, H. (1921) \emph{La végétation malgache}. Musée
Colonial.

\leavevmode\hypertarget{ref-Ploch2012}{}%
Ploch, L. \& Cook, N. (2012) Madagascar's Political Crisis.

\leavevmode\hypertarget{ref-Puyravaud2003}{}%
Puyravaud, J.P. (2003) Standardizing the calculation of the annual rate
of deforestation. \emph{Forest Ecology and Management}, \textbf{177},
593--596.

\leavevmode\hypertarget{ref-Rakotomalala2015}{}%
Rakotomala, F., Rabenandrasana, J., Andriambahiny, J., Rajaonson, R.,
Andriamalala, F., Burren, C., Rakotoarijaona, J., Parany, B., Vaudry,
R., Rakotoniaina, S., Ranaivosoa, R., Rahagalala, P., Randrianary, T. \&
Grinand, C. (2015) Estimation de la déforestation des forêts humides à
Madagascar entre 2005, 2010 et 2013: combinaison multi-date d'images
LANDSAT, utilisation de l'algorithme Random Forest et procédure de
validation. \emph{Revue Française de Photogrammétrie et de
Télédétection}, 11--23.

\leavevmode\hypertarget{ref-Saatchi2011}{}%
Saatchi, S.S., Harris, N.L., Brown, S., Lefsky, M., Mitchard, E.T.A.,
Salas, W., Zutta, B.R., Buermann, W., Lewis, S.L., Hagen, S., Petrova,
S., White, L., Silman, M. \& Morel, A. (2011) Benchmark map of forest
carbon stocks in tropical regions across three continents.
\emph{Proceedings of the National Academy of Sciences}, \textbf{108},
9899--9904.

\leavevmode\hypertarget{ref-Saunders1991}{}%
Saunders, D.A., Hobbs, R.J. \& Margules, C.R. (1991) Biological
Consequences of Ecosystem Fragmentation: A Review. \emph{Conservation
Biology}, \textbf{5}, 18--32.

\leavevmode\hypertarget{ref-Scales2011}{}%
Scales, I.R. (2011) Farming at the Forest Frontier: Land Use and
Landscape Change in Western Madagascar, 1896-2005. \emph{Environment and
History}, \textbf{17}, 499--524.

\leavevmode\hypertarget{ref-Smith2003}{}%
Smith, R.J., Muir, R.D.J., Walpole, M.J., Balmford, A. \&
Leader-Williams, N. (2003) Governance and the loss of biodiversity.
\emph{Nature}, \textbf{426}, 67--70.

\leavevmode\hypertarget{ref-Tidd2001}{}%
Tidd, S.T., Pinder, J. \& Ferguson, G.W. (2001) Deforestation and
habitat loss for the Malagasy flat-tailed tortoise from 1963 through
1993. \emph{Chelonian Conservation and Biology}, \textbf{4}, 59--65.

\leavevmode\hypertarget{ref-Tropek2014}{}%
Tropek, R., Sedláček, O., Beck, J., Keil, P., Musilová, Z., Šímová, I.
\& Storch, D. (2014) Comment on "High-resolution global maps of
21st-century forest cover change". \emph{Science}, \textbf{344},
981--981.

\leavevmode\hypertarget{ref-Tyukavina2015}{}%
Tyukavina, A., Baccini, A., Hansen, M.C., Potapov, P.V., Stehman, S.V.,
Houghton, R.A., Krylov, A.M., Turubanova, S. \& Goetz, S.J. (2015)
Aboveground carbon loss in natural and managed tropical forests from
2000 to 2012. \emph{Environmental Research Letters}, \textbf{10},
074002.

\leavevmode\hypertarget{ref-Tyukavina2017}{}%
Tyukavina, A., Hansen, M.C., Potapov, P.V., Stehman, S.V.,
Smith-Rodriguez, K., Okpa, C. \& Aguilar, R. (2017) Types and rates of
forest disturbance in Brazilian Legal Amazon, 20002013. \emph{Science
Advances}, \textbf{3}.

\leavevmode\hypertarget{ref-UN2015}{}%
United Nations. (2015) \emph{World Population Prospects: The 2015
Revision, Key Findings and Advance Tables. Working Paper No.
ESA/P/WP.241.}

\leavevmode\hypertarget{ref-Verhegghen2016}{}%
Verhegghen, A., Eva, H., Desclée, B. \& Achard, F. (2016) Review and
Combination of Recent Remote Sensing Based Products for Forest Cover
Change Assessments in Cameroon. \emph{International Forestry Review},
\textbf{18}, 14--25.

\leavevmode\hypertarget{ref-Vieilledent2016}{}%
Vieilledent, G., Gardi, O., Grinand, C., Burren, C., Andriamanjato, M.,
Camara, C., Gardner, C.J., Glass, L., Rasolohery, A., Rakoto Ratsimba,
H., Gond, V. \& Rakotoarijaona, J.-R. (2016) Bioclimatic envelope models
predict a decrease in tropical forest carbon stocks with climate change
in Madagascar. \emph{Journal of Ecology}, \textbf{104}, 703--715.

\leavevmode\hypertarget{ref-Vieilledent2013}{}%
Vieilledent, G., Grinand, C. \& Vaudry, R. (2013) Forecasting
deforestation and carbon emissions in tropical developing countries
facing demographic expansion: a case study in Madagascar. \emph{Ecology
and Evolution}, \textbf{3}, 1702--1716.

\leavevmode\hypertarget{ref-Virah-Sawmy2009}{}%
Virah-Sawmy, M. (2009) Ecosystem management in Madagascar during global
change. \emph{Conservation Letters}, \textbf{2}, 163--170.

\leavevmode\hypertarget{ref-Vorontsova2017}{}%
Vorontsova, M.S., Besnard, G., Forest, F., Malakasi, P., Moat, J.,
Clayton, W.D., Ficinski, P., Savva, G.M., Nanjarisoa, O.P., Razanatsoa,
J., Randriatsara, F.O., Kimeu, J.M., Luke, W.R.Q., Kayombo, C. \&
Linder, H.P. (2016) Madagascar's grasses and grasslands: anthropogenic
or natural? \emph{Proceedings of the Royal Society of London B:
Biological Sciences}, \textbf{283}.


\end{document}
