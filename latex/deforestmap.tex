%%%%%%%%%%%%%%%%%%%%%%%%%%%%%%%%%%%%%%%%%%%%%%%%%%%%%%%%%%%%%%%%%%%%%%%%%%%%%%%%%%%%%%%%%%%
%%%%%%%%%%%%%%%%%%%%%%%%%%%%%%%%%%%%%% Preambule %%%%%%%%%%%%%%%%%%%%%%%%%%%%%%%%%%%%%%%%%
%%%%%%%%%%%%%%%%%%%%%%%%%%%%%%%%%%%%%%%%%%%%%%%%%%%%%%%%%%%%%%%%%%%%%%%%%%%%%%%%%%%%%%%%%%

\documentclass[a4paper, 12pt, leqno]{article} %leqno: numéro d'équation à gauche
\pagenumbering{arabic} % choose how to number the pages
\usepackage{a4wide}
\usepackage[utf8]{inputenc} % accents interprétés
\usepackage{graphicx}
%\usepackage{subfig}
%\usepackage[hmargin=2cm, vmargin = 2cm, noheadfoot]{geometry} %% Pour gérer le format des pages
%\usepackage{layout} %% Pour avoir la longueur des marges
\usepackage[round,sort]{natbib} %% Natbib is a popular style for formatting references.
%\usepackage{multibib}
%\newcites{secnm}{Bibliographie} 
%\usepackage{verbatim} % for multiline comments
\usepackage{amssymb} %symbole de maths
\usepackage{amsmath} %idem
%\usepackage{stmaryrd} %% Symbole flèche à l'envers
%\usepackage{amsfonts}
\usepackage[english]{babel} %% Les titres en anglais
\usepackage{array} %% Pour centrer verticalement le contenu d'un tableau, entre autres...
\setcounter{secnumdepth}{4} %% Profondeur des sections, subsections
\usepackage{setspace} %% Gère l'interligne: singlespacing, doublespacing
\usepackage{booktabs}
%\singlespacing
\usepackage{longtable}
\setlength{\LTleft}{-5cm plus 1 fill}
\setlength{\LTright}{-5cm plus 1 fill}
\usepackage[colorlinks=true,citecolor=blue]{hyperref} %% Gère les hyperliens
\usepackage{lineno} %% Numérotation des lignes
\usepackage{caption}
%\linenumbers
\newcommand{\logit}{\text{logit}}
\newcommand{\bs}[1]{\boldsymbol{#1}}
\newcommand{\p}{\text{p}}
\newcommand{\R}{\textnormal{\sffamily\bfseries R}}
\newcommand{\pkg}[1]{{\fontseries{b}\selectfont #1}}

%%%%%%%%%%%%%%%%%%%%%%%%%%%%%%%%%%%%%%%%%%%%%%%%%%%%%%%%%%%%%%%%%%%%%%%%%%%%%%%%%%%%%%%%%%
%%%%%%%%%%%%%%%%%%%%%%%%%%%%%%%%%%%%%% Title %%%%%%%%%%%%%%%%%%%%%%%%%%%%%%%%%%%%%%%%%%%%%
%%%%%%%%%%%%%%%%%%%%%%%%%%%%%%%%%%%%%%%%%%%%%%%%%%%%%%%%%%%%%%%%%%%%%%%%%%%%%%%%%%%%%%%%%%

\title{Combining global tree cover loss data with historical national
  forest-cover maps to look at six decades of deforestation and forest
  fragmentation in Madagascar}
\date{}

%%%%%%%%%%%%%%%%%%%%%%%%%%%%%%%%%%%%%%%%%%%%%%%%%%%%%%%%%%%%%%%%%%%%%%%%%%%%%%%%%%%%%%%%%%
%%%%%%%%%%%%%%%%%%%%%%%%%%%%%%%%%%%%%% Document %%%%%%%%%%%%%%%%%%%%%%%%%%%%%%%%%%%%%%%%%%
%%%%%%%%%%%%%%%%%%%%%%%%%%%%%%%%%%%%%%%%%%%%%%%%%%%%%%%%%%%%%%%%%%%%%%%%%%%%%%%%%%%%%%%%%%

\begin{document}
\maketitle

\vspace{-1cm}
\begin{center}
%%{\large
  Ghislain Vieilledent$^{1,2,\star}$\hspace{1cm}
  Clovis Grinand$^{3}$ \\
  \vspace{0.25cm}
  Fety A. Rakotomalala$^{3}$ \hspace{1cm}
  Rija Ranaivosoa$^{4}$ \hspace{1cm}
  Jean-Roger Rakotoarijaona$^{4}$ \\
  \vspace{0.25cm}
  Thomas F. Allnutt$^{5,6}$\hspace{1cm}and\hspace{1cm}
  Frédéric Achard$^{1}$
%%}
\end{center}

\vspace{0.3cm}

{\small
  \begin{flushleft}  
    $[1]$ \textbf{Joint Research Center of the European Commission} -- Bio-economy Unit, I-21027 Ispra (VA), ITALY\\
    $[2]$ \textbf{Cirad} -- UPR Forêts et Sociétés, F-34398 Montpellier, FRANCE\\
    $[3]$ \textbf{ETC Terra}, F-75020 Paris, FRANCE\\
    $[4]$ \textbf{Office National pour l'Environnement}, 101 Antananarivo, MADAGASCAR\\
    $[5]$ \textbf{Wildlife Conservation Society}, 101 Antananarivo, MADAGASCAR\\
    $[6]$ \textbf{GreenInfo Network}, Oakland, California, USA\\
    ~\\
    $[\star]$ \textbf{Corresponding author:}
    \textbackslash{E-mail}:~ghislain.vieilledent@cirad.fr
    \textbackslash{Phone}:~+39.033.278.3516\\
  \end{flushleft}}

\vspace{0.3cm}

\begin{center}
  \textbf{Running headline:} \emph{Six decades of deforestation in Madagascar}
\end{center}

\newpage

\linenumbers
\doublespacing

\newcommand{\keywords}[1]{\par\noindent
{\small{\em Keywords\/}: #1}}

\begin{abstract}

  1. The island of Madagascar has an unparalleled biodiversity, mainly
  located in the tropical forests of the island, which is highly
  threatened by anthropogenic deforestation. Scattered forest maps
  from past studies at national level with substantial gaps (due to
  presence of cloud cover on satellite imagery) prevent the analyzis
  of long-term deforestation trends in Madagascar.

  2. In this study, we propose a new approach combining historical
  (1953-2000) national forest-cover maps with recent (2001-2014)
  global annual tree cover loss data to look at six decades
  (1953-2014) of deforestation and forest fragmentation in
  Madagascar. We produced new forest-cover maps at 30~m resolution over
  the full territory of Madagascar for the year 1990, and annually
  from 2000 to 2014.

  3. We estimated that Madagascar has lost 44\% of its natural forest
  cover over the period 1953-2014 (including 37\% over the period
  1973-2014). Natural forests cover 8.9 Mha in 2014 (15\% of the
  national territory) which are divided into 4.4 Mha (50\%) of moist
  forests, 2.6 Mha (29\%) of dry forests, 1.7 Mha of spiny forests
  (19\%) and 177,000 ha (2\%) of mangroves. Since 2005, the annual
  deforestation rate has progressively increased in Madagascar to
  reach 99,000 ha/yr during 2010-2014 (corresponding to a rate of
  1.08\%/yr). This increase is probably due to rapid population growth
  (close to 3\%/yr) and to poor law enforcement in the country. Around
  half of the forest (46\%) is now located at less than 100m from the
  forest edge.

  4. \emph{Policy implications}: Accurate forest-cover change maps can
  be used to assess the effectiveness of past and current conservation
  programs and implement new strategies for the future. In particular,
  forest maps and estimates can be used in the framework of the REDD+
  (``Reducing Emissions from Deforestation and Forest Degradation'')
  initiative and for optimizing the current protected area network.

\vspace{0.5cm}

\keywords{biodiversity, climate-change, deforestation, Madagascar, tropical forest}

\end{abstract}

\newpage

\section{Introduction}
\label{introduction}

Separated from the African continent and the Indian plate about 165
and 88 million years ago respectively \citep{Ali2008}, the flora and
fauna of Madagascar followed its own evolutionary path. Isolation
combined with a high number of micro-habitats \citep{Pearson2009} has
led to Madagascar's exceptional biodiversity both in term of number of
species and endemism in many taxonomic groups \citep{Crottini2012,
  Goodman2005}. Most of the biodiversity in Madagascar is
concentrated in the tropical forests of the island which can be
divided into four types: the moist forest in the East, the dry forest
in the West, the spiny forest in the South and the mangroves on the
West coast \citep{Vieilledent2016}. This unparalleled biodiversity is
severely threatened by deforestation \citep{Harper2007,
  Vieilledent2013} associated with human activities such as slash-and-burn
agriculture and pasture \citep{Scales2011}. Tropical forests in
Madagascar also store a large amount of carbon \citep{Vieilledent2016}
and high rates of deforestation in Madagascar are responsible for
large CO$_2$ emissions in the atmosphere
\citep{Achard2014}. Deforestation threatens species survival by
directly reducing their available habitat \citep{Brooks2002,
  Tidd2001}. Forest fragmentation can also lead to species extinction
by isolating populations from each other and creating forest patches
too small to maintain viable populations
\citep{Saunders1991}. Fragmentation also increases forest edge where
ecological conditions (such as air temperature, light intensity and
air moisture) can be dramatically modified, with consequences on the
abundance and distribution of species \citep{Murcia1995}. Forest
fragmentation can also have substantial effects on forest carbon
storage capacity, as carbon stocks are much lower at the forest edge
than under a closed canopy \citep{Brinck2017}. Moreover, forest
carbon stocks vary spatially due to climate or soil factors
\citep{Saatchi2011, Vieilledent2016}. As a consequence, accurate and
spatially explicit maps of forest-cover and forest-cover change are
necessary to monitor biodiversity loss and carbon emissions from
deforestation and forest fragmentation, assess the efficiency of
present conservation strategies \citep{Eklund2016}, and implement new
strategies for the future \citep{Vieilledent2013,
  Vieilledent2016}. Simple time-series of forest-cover estimates, such
as those provided by the FAO Forest Resource Assessment report
\citep{Keenan2015} are not sufficient.

Unfortunately, accurate and exhaustive forest-cover maps are not
available for Madagascar for the last fifteen years (2000-2015).
\citet{Harper2007} produced maps of forest cover and forest cover
changes over Madagascar for the years \emph{c.}~1953, \emph{c.}~1973,
1990 and 2000. The \emph{c.}~1953 forest map was derived from the
visual interpretation of aerial photography at coarse scale
(1/1,000,000). Forest maps for the years \emph{c.}~1973, 1990, and
2000 were obtained from supervised classification of Landsat satellite
images at 60~m resolution (for the year 1973) or 30~m resolution (for
years 1990 and 2000) and can be used to derive more accurate estimates
of forest cover (89.5\% accuracy reported for the forest/non-forest
map of year 2000). Nonetheless, maps provided by \citet{Harper2007}
are not exhaustive (due to the presence of clouds in the satellite
imagery), e.g.~11 244~km2 are mapped as unknown cover type for the
year 2000. Using a similar supervised classification approach as in
\citet{Harper2007}, more recent maps have been produced for the
periods 2000-2005-2010 by national institutions, with the technical
support of international environmental NGOs \citep{MEFT2009,
  ONE2013}. Another set of recent forest-cover maps using an advanced
statistical tool for classification, the Random Forest classifier
\citep{Grinand2013, Rakotomalala2015}, was produced for the periods
2005-2010-2013 \citep{ONE2015}. However, these maps are either too old
to give recent estimates of deforestation \citep{MEFT2009, ONE2013},
include large areas of missing information due to images with high
percentage of cloud cover \citep{ONE2013}, or show large
mis-classification in specific areas, especially in the dry and spiny
forest domain for which the spectral answer has a strong seasonal
behavior due to the deciduousness of such forests (overall accuracy is
lower than 0.8 for the dry and spiny forests for the maps produced by
\citet{ONE2015}). Moreover, the production of such forest maps from a
supervised classification approach requires significant resources,
especially regarding the image selection step (required to minimize
cloud cover) and the training step (visual interpretation of a large
number of polygons needed to train the classification algorithm)
\citep{Rakotomalala2015}. Most of this work of image selection and
visual interpretation would need to be repeated to produce new forest
maps in the future using a similar approach.

Global forest or tree cover products have also been published recently
and can be tested at the national scale for
Madagascar. \citet{Kim2014} produced a global forest-cover change map
from 1990 to 2000 (derived from Landsat imagery). This product was
updated to cover the period 1975-2005
(\url{http://glcf.umd.edu/data/landsatFCC/}) but forest-cover maps
after 2005 were not produced. Moreover, the approach used in
\citet{Kim2014} did not accurately map the forests in the dry and
spiny ecosystems of Madagascar (see Fig. 8 in \citet{Kim2014}).
\citet{Hansen2013} mapped tree cover percentage, annual forest loss
and forest gain from 2000 to 2012 at global scale at 30 m
resolution. This product has since been updated and is now available
up to the year 2014 \citep{Hansen2013}. To map forest cover from the
\citet{Hansen2013} product, a tree cover threshold must be selected
(that defines forest cover). Selecting such a threshold is not
straightforward as the accuracy of the global tree cover map strongly
varies between forest types, and is substantially lower for dry
forests than for moist forests \citep{Bastin2017}. Moreover, the
\citet{Hansen2013} product does not provide information on
land-use. In particular the global tree cover map does not separate
tree plantations such as oil palm or eucalyptus plantations from
natural forests \citep{Tropek2014}. Thus, the global tree cover map
from \citet{Hansen2013} cannot be used alone to produce a map of
forest cover \citep{Tyukavina2017}. In complement to the tree cover
percentage provided in \citet{Hansen2013}, a layer of annual tree
cover loss is also provided (i.e.~complete loss of tree cover from a
value higher than 10\% to zero) for the period 2001-2014.

In this study, we present a simple approach which combines the maps
from \citet{Harper2007} and products from \citet{Hansen2013} to derive
annual wall-to-wall forest-cover change maps over the period 2000-2014
for Madagascar. We use the forest-cover map provided by
\citet{Harper2007} for the year 2000 (defining the land-use) with the
tree cover loss product provided by \citet{Hansen2013} that we apply
only inside forest areas identified by \citet{Harper2007}. Similar to
the approach of \citet{Harper2007}, we also assess trends in
deforestation rates and forest fragmentation from \emph{c.} 1953 to 2014. The
approach described in this study can help assess the effectiveness of
current conservation strategies, and assist the implementation of
future strategies. Our approach could be easily extended to other
tropical countries that have at least one forest-cover map between
2000 and 2014. This approach can easily be repeated in the future when
the \citet{Hansen2013} products are updated.

\newpage

\section{Materials and Methods}
\label{materials-and-methods}

\subsection{Creation of new forest-cover maps of Madagascar from
1953 to 2014}

We produced annual forest/non-forest maps at 30~m resolution for the
full territory of Madagascar for the period 2000-2014 by combining the
forest map of year 2000 from \citet{Harper2007}, and the tree cover
percentage and annual forest cover loss maps over the period 2000-2014
from \citet{Hansen2013}. The 2000 Harper's forest map includes 208,000
ha of unclassified areas due to the presence of clouds on satellite
images, mostly (88\%) within the moist forest domain which covered
4.17 Mha in total in 2000. To provide a label (forest or non-forest)
to these unclassified pixels, we used the 2000 tree cover percentage
map of \citet{Hansen2013} by selecting a threshold of 75\% tree cover
to define forest cover as recommended by other studies for the moist
domain \citep{Achard2014}. We thus obtained a forest-cover map for the
year 2000 covering the full territory of Madagascar. We then combined
this forest-cover map of the year 2000 with the annual tree cover loss
maps from 2001 to 2014 provided by \citet{Hansen2013} to create annual
forest-cover maps from 2001 to 2014 at 30~m resolution. We also
completed the Harper's forest map of year 1990 by filling unclassified
areas (due to the presence of clouds on satellite images) using our
forest-cover map of year 2000. To do so, we assumed that if forest was
present in 2000, the pixel was also forested in 1990. The remaining
unclassified pixels were limited to a relatively small total area of
\emph{c.} 8,000 ha. We labeled these residual pixels as non-forest as
for the year 2000. Similarly we completed the Harper's forest map of
year 1973 by filling unclassified areas using our forest-cover map of
the year 1990 assuming that if forest was present in 1990, it was also
present in 1973. Contrary to the year 1990, the remaining unclassified
pixels for year 1973 corresponded to a significant total area of 3.32
million ha. We also reprojected the forest-cover map of year 1953 to a
common projection in order to compare the forest-cover area in 1953
with forest-cover areas at the following dates. This map was produced
by scanning a paper map derived from aerial photos, and thus could not
be perfectly aligned with the other maps produced through digital
processing of satellite imagery \citep{Harper2007}. Finally for all
forest-cover maps from 1973, the isolated single non-forest pixels
(i.e.~fully surrounded by forest pixels) were removed, assuming that
single non-forest pixels inside a forest patch were not corresponding
to deforestation (they might correspond to selective logging
activities). This allowed us to avoid counting very small scale events
(\textless{}0.1 ha such as selective logging) as forest
fragmentation. All the resulting maps are freely available at
\url{https://bioscenemada.cirad.fr/forestmaps}.

\subsection{Computing forest-cover areas and deforestation
rates}

From these new forest-cover maps, we calculated the total forest-cover
area for seven available years (1953-1973-1990-2000-2010-2005-2014),
and the annual deforested area and annual deforestation rate for the
corresponding six time periods between 1953 and 2014. The annual
deforestation rates were calculated as follows \citep{Puyravaud2003,
  Vieilledent2013}:

\begin{equation*}
  \theta = 100 \times [1-(1-(F_{t_2}-F_{t_1})/F_{t_1})^{(1/(t_2-t_1))}
\end{equation*}

where $\theta$ is the annual deforestation rate (in \%/yr),
$F_{t_2}$ and $F_{t_1}$ are the forest cover free of clouds at both
dates $t_2$ and $t_1$, and $t_2-t_1$ is the time-interval (in
years) between the two dates.

Because of the large unclassified area (3.32 million ha) in 1973, the
annual deforestation areas and rates for the two periods 1953-1973 and
1973-1990 are only indicative estimates. For these two periods the
annual deforestation rates are computed as the ratio
$(F_{t_2}-F_{t_1})/F_{t_1}$ considering only the mapped forest pixels.
Area and rate estimates are produced at the national scale and for the
four forest ecosystems present in Madagascar: moist forest in the
East, dry forest in the West, spiny forest in the South, and mangroves
on the Western coast (Fig.~\ref{fig:ecoregion}). To define the forest
domains, we used a map from the MEFT (\emph{``Ministère de
  l'Environnement et des Forêts à Madagascar''}) with the boundaries
of the four ecoregions in Madagascar. Ecoregions were defined on the
basis of climatic and vegetation criteria using the climate
classification by \citet{Cornet1974} and the vegetation classification
from the 1996 IEFN national forest inventory \citep{IEFN1996}. Because
mangrove forests are highly dynamic ecosystems that can expand or
contract on decadal scales depending on changes in environmental
factors \citep{Armitage2015}, a fixed delimitation of the mangrove
ecoregion on six decades might not be fully appropriate. As a
consequence, our estimates of the forest-cover and deforestation rates
for mangroves in Madagascar must be considered with this limitation.

\subsection{Comparing our forest-cover and deforestation rate
estimates with previous studies}

We compared our estimates of forest-cover and deforestation rates with
estimates from the three existing studies at the national scale for
Madagascar: (i) \citep{Harper2007}, (ii) \citep{MEFT2009} and (iii)
\citep{ONE2015}. \citet{Harper2007} provides forest-cover and
deforestation estimates for the periods
c. 1953-c. 1973-1990-2000. MEFT, USAID, and CI (2009) provides
estimates for the periods 1990-2000-2005 and ONE, DGF, MNP, WCS, and
Etc Terra (2015) provides estimates for the periods 2005-2010-2013. To
compare our forest-cover and deforestation estimates over the same
time periods, we consider an additional time-period in our study
(2010-2013) by creating an extra forest-cover map for the year
2013. We computed the Pearson's correlation coefficient and the root
mean square error (RMSE) between our forest-cover estimates and
forest-cover estimates from previous studies for all the dates and
forest types (including also the total forest cover estimates). For
previous studies, the computation of annual deforestation rates (in
\%/yr) is not always detailed and might slightly differ from one study
to another \citep[see][]{Puyravaud2003}. \citet{Harper2007} also
provide total deforested areas for the two periods 1973-1990 and
1990-2000. We converted these values into annual deforested area
estimates. When annual deforested areas were not reported (for
1953-1973 in \citet{Harper2007} and in \citet{MEFT2009} and
\citet{ONE2015}), we computed them from the forest-cover estimates in
each study. These estimates cannot be corrected from the potential
bias due to the presence of residual clouds. Forest-cover and
deforestation rates were then compared between all studies for the
whole of Madagascar and the four ecoregions. The same ecoregion
boundaries as in our study were used in \citet{ONE2015} but this was
not the case for \citet{Harper2007} and \citet{MEFT2009}, which can
explain part of the differences between the estimates.

\subsection{Fragmentation}

We also conducted an analysis of changes in forest fragmentation for
the years 1953, 1973, 1990, 2000, 2005, 2010 and 2014. We applied the
method developed by \citet{Riitters2000} which uses a moving window to
characterize the fragmentation around each forested
pixel. Computations were done using the function \texttt{r.forestfrag}
of the GRASS GIS software \citep{Neteler2008}. Six categories of
fragmentation were identified from the amount of forest and its
occurrence as adjacent forest pixels: ``interior'', ``perforated'',
``edge'', ``transitional'', ``patch'', and ``undetermined''. We used a
moving window of 7x7 pixels (4.4 ha). Using this window size, forest
edge had a width of about 90m \citep{Riitters2000}. The ``interior''
category can be interpreted as the most intact forest
\citep{Potapov2017}. The ``patch'' and ``transitional'' categories
correspond to isolated small forest patches. We reported the area of
forest in each fragmentation category for the six years and analyzed
the dynamics of fragmentation over the six decades. We also computed
the distance to forest edge for all forest pixels for the years 1953,
1973, 1990, 2000, 2005, 2010 and 2014. For that, we used the function
\texttt{gdal\_proximity.py} of the GDAL software
(\url{http://www.gdal.org/}). We computed the mean and 90\% quantiles
(5\% and 95\%) of the distance to forest edge and looked at the
evolution of these values with time.

\newpage

\section{Results}
\label{results}

\subsection{Dynamics of forest cover and deforestation
intensity}

Natural forests in Madagascar covered 16.0 Mha in 1953, about 27\% of
the national territory of 587,041 km2. In 2014, the forest cover
dropped to 8.9 Mha, corresponding to about 15\% of the national
territory (Fig.~\ref{fig:fcc} and
Tab.~\ref{tab:forest_cover}). Madagascar has lost 44\% and 37\% of its
natural forests between 1953 and 2014, and between 1973 and 2014
respectively (Fig.~\ref{fig:fcc} and Tab.~\ref{tab:forest_cover}). In
2014 the remaining 8.9 Mha of natural forest were distributed as: 4.4
Mha of moist forest (50\% of total forest cover), 2.6 Mha of dry
forest (29\%), 1.7 Mha of spiny forest (19\%) and 0.18 Mha (2\%) of
mangrove forest (Fig.~\ref{fig:ecoregion} and
Tab.~\ref{tab:comp_forest}). Regarding the deforestation trend, we
observed a progressive decrease of the deforestation rate after 1990
from 205,000 ha/yr (1.63\%/yr) over the period 1973-1990 to 44,300
ha/yr (0.43\%/yr) over the period 2000-2005
(Tab.~\ref{tab:forest_cover}). Then from 2005, the deforestation rate
has progressively increased and has more than doubled over the period
2010-2014 (98,700 ha/yr, 1.08\%/yr) compared to 2000-2005
(Tab.~\ref{tab:forest_cover}). The deforestation trend characterized
by a progressive decrease of the deforestation rate over the period
1990-2005 and a progressive increase of the deforestation after 2005
is valid for all four ecoregions (Tab.~\ref{tab:comp_defor}), with the
exception of the spiny forest domain for which the deforestation rate
during the period 2010-2013 was lower than during 2005-2010
(Tab.~\ref{tab:comp_defor}).

\subsection{Comparison with previous forest-cover change studies
in Madagascar}

Forest-cover maps provided by previous studies over Madagascar were
not exhaustive (unclassified areas) due to the presence of clouds on
satellite images used to produce such maps. In \citet{Harper2007}, the
maps of years 1990 and 2000 include 0.5 and 1.12 Mha of unknown cover
type respectively. Proportions of unclassified areas are not reported
in the two other existing studies by \citet{MEFT2009} and
\citet{ONE2015}. With our approach, we produced wall to wall
forest-cover change maps from 1990 to 2014 for the full territory of
Madagascar (Tab.~\ref{tab:forest_cover}). This allowed us to produce
more robust estimates of forest-cover and deforestation rates over
this period. Our forest-cover estimates over the period 1953-2013
(considering forest cover estimates at national level and by
ecoregions for all the available dates) were well correlated
(Pearson's correlation coefficient $=$ 0.99) to estimates from the
three previous studies (Tab.~\ref{tab:comp_forest}) with a RMSE of
300,000 ha (6\% of the mean forest cover of 4.8 Mha when considering
all dates and forest types together). These small differences can be
partly attributed to differences in ecoregion boundaries. Despite
significant differences in deforestation estimates
(Tab.~\ref{tab:comp_defor}), a similar deforestation trend was
observed across studies with a decrease of deforestation rates over
the period 1990-2005, followed by a progressive increase of the
deforestation after 2005.

\subsection{Evolution of forest fragmentation with time}

In parallel to the dynamics of deforestation, forest fragmentation has
progressively increased since 1953 in Madagascar. We observed a
continuous decrease of the mean distance to forest edge from 1953 to
2014 in Madagascar. The mean distance to forest edge has decreased to
\emph{c.} 300~m in 2014 while it was previously \emph{c.} 1.5~km in 1973
(Fig.~\ref{fig:dist_edge}). Moreover, a large proportion (73\%) of
the forest was located at a distance greater than 100 m in 1973, while
almost half of the forest (46\%) was at a distance lower than 100 m
from forest edge in 2014 (Fig.~\ref{fig:dist_edge}). The percentage of
forest that can be considered intact in Madagascar has continuously
decreased since 1953. The percentage of forest belonging to the
``interior'' category (most intact forests) has fallen from 68\% in
1973 to 50\% in 2014. In 2014, more than 16\% of the forest belonged
to the ``patch'' and ``transitional'' categories (isolated small
forest patches) compared to 9.5\% in 1973 (Tab.~\ref{tab:frag}).

\newpage

\section{Discussion}
\label{discussion}

\subsection{Benefits of the combined use of recent global annual
  tree cover loss data with historical national forest-cover maps}

In this study, we combined recent (2001-2014) global annual tree cover
loss data \citep{Hansen2013} with historical (1953-2000) national
forest-cover maps \citep{Harper2007} to look at six decades
(1953-2014) of deforestation and forest fragmentation in
Madagascar. We produced annual forest-cover maps at 30~m resolution
covering Madagascar for the period 2000 to 2014. Our study extends the
forest-cover monitoring on a six decades period (from 1953 to 2014)
while harmonizing the data from previous studies \citep{Harper2007,
  MEFT2009, ONE2015}. We propose a generic approach to solve the
problem of forest definition which is needed to transform the 2000
global tree cover dataset from \citet{Hansen2013} into a
forest/non-forest map \citep{Tropek2014}. We propose to use a
historical national forest-cover map, based on a national forest
definition, as a forest cover mask. This approach could be easily
extended to other regions or countries for which an accurate
forest-cover map is available at any date within the period 2000-2014,
but preferably at the beginning of the period to profit from the full
record and derive long-term estimates of deforestation. Moreover, this
approach can be repeated in the future if and when the global tree
cover product is updated. We have made the R/GRASS code used for this
study freely available in a GitHub repository (see Data availability
statement) to facilitate application to other study areas or repeat
the analysis in the future for Madagascar.

The accuracy of the derived forest-cover change maps depends directly
on the accuracies of the historical forest-cover maps and the tree
cover loss dataset. The reported global accuracy of the tree cover
loss dataset is 99.6\% (see Tab. S5 in \citet{Hansen2013}).
\citet{Verhegghen2016} have compared deforestation estimates derived
from the global tree cover loss dataset \citep{Hansen2013} with
results derived from semi-automated supervised classification of
Landsat satellite images \citep{Achard2014} for six countries in
Central Africa and they found a good agreement between these two sets
of estimates. Consistent with \citet{Harper2007}, we did not consider
potential forest regrowth in Madagascar (although \citet{Hansen2013}
provided a tree cover gains layer for the period 2001-2013) for
several reasons. First, the tree gain layer of \citet{Hansen2013}
includes and catches more easily tree plantations than natural forest
regrowth \citep{Tropek2014}. Second, there is little evidence of
natural forest regeneration in Madagascar \citep{Grouzis2001,
 Harper2007}. This can be explained by several ecological processes
following burning practice such as soil erosion \citep{Grinand2017}
and reduced seed bank due to fire and soil loss
\citep{Grouzis2001}. Moreover, in areas where forest regeneration is
ecologically possible, young forest regrowth are more easily re-burnt
for agriculture and pasture. Third, young secondary forests provide
more limited ecosystem services compared to old-growth natural forests
in terms of biodiversity and carbon storage.

\subsection{Dynamics of forest-cover in Madagascar from 1953 to 2014}

We estimated that natural forests in Madagascar cover 8.9 Mha in 2014
(corresponding to 15\% of the country) and that Madagascar has lost
44\% of its natural forest since 1953 (37\% since 1973). There is
ongoing scientific debate about the extent of the ``original'' forest
cover in Madagascar, and the extent to which humans have altered the
natural forest landscapes since their large-scale settlement around
800 CE \citep{Burns2016, Cox2012}. Early French naturalists stated
that the full island was originally covered by forest
\citep{Humbert1927, Perrier1921}, leading to the common statement that
90\% of the natural forests have disappeared since the arrival of
humans on the island \citep{Kull2000}. More recent studies
counter-balanced that point of view saying that extensive areas of
grassland existed in Madagascar long before human arrival and were
determined by climate, natural grazing and other natural factors
\citep{Vorontsova2017, Virah-Sawmy2009}. Other authors have questioned
the entire narrative of extensive alteration of the landscape by early
human activity which, through legislation, has severe consequences on
local people \citep{Klein2002, Kull2000}. Whatever the original
proportion of natural forests and grasslands in Madagascar, our
results demonstrate that human activities since the 1950s have
profoundly impacted the natural tropical forests and that conservation
and development programs in Madagascar have failed to stop
deforestation in the recent years. Deforestation has strong
consequences on biodiversity and carbon emissions in
Madagascar. Around 90\% of Madagascar's species are forest dependent
\citep{Allnutt2008, Goodman2005} and \citet{Allnutt2008} estimated
that deforestation between 1953 and 2000 led to an extinction of 9\%
of the species. The additional deforestation we observed over the
period 2000-2014 (around 1Mha of natural forest) worsen this
result. Regarding carbon emissions, using the 2010 aboveground forest
carbon map by \citet{Vieilledent2016}, we estimated that deforestation
on the period 2010-2014 has led to 40.2 Mt C of carbon emissions in
the atmosphere (10 Mt C /yr) and that the remaining aboveground forest
carbon stock in 2014 is 832.8 Mt C. Associated to deforestation, we
showed that the remaining forests of Madagascar are highly fragmented
with 46\% of the forest being at less than 100m of the forest
edge. Small forest fragments do not allow to maintain viable
populations and ``edge effects'' at forest/non-forest interfaces have
impacts on both carbon emissions \citep{Brinck2017} and biodiversity
loss \citep{Gibson2013, Murcia1995}.

\subsection{Deforestation trend and impacts on conservation and
  development policies}

In our study, we have shown that the progressive decrease of the
deforestation rate on the period 1990-2005 was followed by a
continuous increase in the deforestation rate on the period
2005-2014. In particular, we showed that deforestation rate has more
than doubled on the period 2010-2014 compared to 2000-2005. Our
results are confirmed by previous studies \citep{Harper2007, MEFT2009,
  ONE2015} despite differences in the methodologies regarding (i)
forest definition (associated to independent visual interpretations of
observation polygons to train the classifier), (ii) classification
algorithms, (iii) deforestation rate computation method, and (iv)
correction for the presence of clouds. Our deforestation rate
estimates from 1990 to 2014 have been computed from wall to wall maps
at 30~m resolution and can be considered more accurate in comparison
with estimates from these previous studies. Our forest-cover and
deforestation rate estimates can be used as source of information for
the next FAO Forest Resources Assessment project
\citep{Keenan2015}. Current rates of deforestation can also be used to
build reference scenarios for deforestation in Madagascar and
contribute to the implementation of deforestation mitigation
activities in the framework of REDD+ \citep{Olander2008}.

The increase of deforestation rates after 2005 can be explained by
population growth and political instability in the country. Nearly
90\% of Madagascar's population relies on biomass for their daily
energy needs \citep{Minten2013} and the link between population size
and deforestation has previously been demonstrated in Madagascar
\citep{Vieilledent2013, Gorenflo2011}. With a mean demographic growth
rate of about 2.8\%/yr and a population which has increased from 16 to
24 million people on the period 2000-2015 \citep{UN2015}, the
increasing demand in wood-fuel and space for agriculture is likely to
explain the increase in deforestation rates. The political crisis of
2009 \citep{Ploch2012}, followed by several years of political
instability and weak governance could also explain the increase in the
deforestation rate observed on the period 2005-2014
\citep{Smith2003}. These results show that despite the conservation
policy in Madagascar \citep{Freudenberger2010}, deforestation has
dramatically increased at the national level since 2005. Results of
this study, including recent spatially explicit forest-cover change
maps and forest-cover estimates, should help implement new
conservation strategies to save Madagascar natural tropical forests
and their unique biodiversity.

\newpage

\section{Author's contribution}
\label{authors-contribution}

All authors conceived the ideas and designed methodology; GV analysed
the data and wrote the {\R}/GRASS script; GV drafted the manuscript. All
authors contributed critically to the drafts and gave final approval for
publication.

\section{Acknowledgements}
\label{acknowledgements}

This study is part of the Cirad's BioSceneMada project
(\url{https://bioscenemada.cirad.fr}) and the Joint Research Center's
ReCaREDD project (\url{http://forobs.jrc.ec.europa.eu/recaredd}). The
BioSceneMada project is funded by FRB (Fondation pour la Recherche sur
la Biodiversité) and the FFEM (Fond Français pour l'Environnement
Mondial) under the project agreement AAP-SCEN-2013 I. The ReCaREDD
project is funded by the European Commission.

\section{Data accessibility}
\label{data-accessibility}

All the data and codes used for this study are made publicly available
in the \texttt{deforestmap} GitHub repository
(\url{https://github.com/ghislainv/deforestmap.git}). The results are
fully reproducible running the {\R} script \texttt{deforestmap.R}
located inside the \texttt{deforestmap} repository.

\newpage
\singlespacing

%%\bibliographystyle{bib/jae}
%%\bibliography{bib/biblio}

\section{References}

\renewcommand{\bibsection}{} %% Remove section name for bibliography
\begin{thebibliography}{55}
\providecommand{\natexlab}[1]{#1}

\bibitem[{Achard \emph{et~al.}(2014)Achard, Beuchle, Mayaux, Stibig, Bodart,
  Brink, Carboni, Desclée, Donnay, Eva, Lupi, Raši, Seliger \&
  Simonetti}]{Achard2014}
Achard, F., Beuchle, R., Mayaux, P., Stibig, H.J., Bodart, C., Brink, A.,
  Carboni, S., Desclée, B., Donnay, F., Eva, H.D., Lupi, A., Raši, R.,
  Seliger, R. \& Simonetti, D. (2014) Determination of tropical deforestation
  rates and related carbon losses from 1990 to 2010.
\newblock \emph{Global Change Biology}, \textbf{20}, 2540--2554.

\bibitem[{Ali \& Aitchison(2008)}]{Ali2008}
Ali, J.R. \& Aitchison, J.C. (2008) {Gondwana to Asia: Plate tectonics,
  paleogeography and the biological connectivity of the Indian sub-continent
  from the Middle Jurassic through latest Eocene (166--35 Ma)}.
\newblock \emph{Earth-Science Reviews}, \textbf{88}, 145--166.

\bibitem[{Allnutt \emph{et~al.}(2008)Allnutt, Ferrier, Manion, Powell,
  Ricketts, Fisher, Harper, Irwin, Kremen, Labat, Lees, Pearce \&
  Rakotondrainibe}]{Allnutt2008}
Allnutt, T.F., Ferrier, S., Manion, G., Powell, G.V.N., Ricketts, T.H., Fisher,
  B.L., Harper, G.J., Irwin, M.E., Kremen, C., Labat, J.N., Lees, D.C., Pearce,
  T.A. \& Rakotondrainibe, F. (2008) {A method for quantifying biodiversity
  loss and its application to a 50-year record of deforestation across
  Madagascar}.
\newblock \emph{Conservation Letters}, \textbf{1}, 173--181.

\bibitem[{Armitage \emph{et~al.}(2015)Armitage, Highfield, Brody \&
  Louchouarn}]{Armitage2015}
Armitage, A.R., Highfield, W.E., Brody, S.D. \& Louchouarn, P. (2015) {The
  contribution of mangrove expansion to salt marsh loss on the Texas Gulf
  Coast}.
\newblock \emph{PloS One}, \textbf{10}, e0125404.

\bibitem[{Bastin \emph{et~al.}(2017)Bastin, Berrahmouni, Grainger, Maniatis,
  Mollicone, Moore, Patriarca, Picard, Sparrow, Abraham, Aloui, Atesoglu,
  Attore, Bass{\"u}ll{\"u}, Bey, Garzuglia, Garc{\'\i}a-Montero, Groot, Guerin,
  Laestadius, Lowe, Mamane, Marchi, Patterson, Rezende, Ricci, Salcedo, Diaz,
  Stolle, Surappaeva \& Castro}]{Bastin2017}
Bastin, J.F., Berrahmouni, N., Grainger, A., Maniatis, D., Mollicone, D.,
  Moore, R., Patriarca, C., Picard, N., Sparrow, B., Abraham, E.M., Aloui, K.,
  Atesoglu, A., Attore, F., Bass{\"u}ll{\"u}, {\c C}., Bey, A., Garzuglia, M.,
  Garc{\'\i}a-Montero, L.G., Groot, N., Guerin, G., Laestadius, L., Lowe, A.J.,
  Mamane, B., Marchi, G., Patterson, P., Rezende, M., Ricci, S., Salcedo, I.,
  Diaz, A.S.P., Stolle, F., Surappaeva, V. \& Castro, R. (2017) The extent of
  forest in dryland biomes.
\newblock \emph{Science}, \textbf{356}, 635--638.

\bibitem[{Brinck \emph{et~al.}(2017)Brinck, Fischer, Groeneveld, Lehmann,
  Dantas De~Paula, Pütz, Sexton, Song \& Huth}]{Brinck2017}
Brinck, K., Fischer, R., Groeneveld, J., Lehmann, S., Dantas De~Paula, M.,
  Pütz, S., Sexton, J.O., Song, D. \& Huth, A. (2017) High resolution analysis
  of tropical forest fragmentation and its impact on the global carbon cycle.
\newblock \emph{Nature Communications}, \textbf{8}, 14855.

\bibitem[{Brooks \emph{et~al.}(2002)Brooks, Mittermeier, Mittermeier,
  da~Fonseca, Rylands, Konstant, Flick, Pilgrim, Oldfield, Magin \&
  Hilton-Taylor}]{Brooks2002}
Brooks, T.M., Mittermeier, R.A., Mittermeier, C.G., da~Fonseca, G.A.B.,
  Rylands, A.B., Konstant, W.R., Flick, P., Pilgrim, J., Oldfield, S., Magin,
  G. \& Hilton-Taylor, C. (2002) Habitat loss and extinction in the hotspots of
  biodiversity.
\newblock \emph{Conservation Biology}, \textbf{16}, 909--923.

\bibitem[{Burns \emph{et~al.}(2016)Burns, Godfrey, Faina, McGee, Hardt,
  Ranivoharimanana \& Randrianasy}]{Burns2016}
Burns, S.J., Godfrey, L.R., Faina, P., McGee, D., Hardt, B., Ranivoharimanana,
  L. \& Randrianasy, J. (2016) {Rapid human-induced landscape transformation in
  Madagascar at the end of the first millennium of the Common Era}.
\newblock \emph{Quaternary Science Reviews}, \textbf{134}, 92 -- 99.

\bibitem[{Cornet(1974)}]{Cornet1974}
Cornet, A. (1974) {Essai de cartographie bioclimatique à Madagascar}.
\newblock , Orstom.

\bibitem[{Cox \emph{et~al.}(2012)Cox, Nelson, Tumonggor, Ricaut \&
  Sudoyo}]{Cox2012}
Cox, M.P., Nelson, M.G., Tumonggor, M.K., Ricaut, F.X. \& Sudoyo, H. (2012) {A
  small cohort of Island Southeast Asian women founded Madagascar}.
\newblock \emph{Proceedings of the Royal Society B: Biological Sciences},
  \textbf{279}, 2761--2768.

\bibitem[{Crottini \emph{et~al.}(2012)Crottini, Madsen, Poux, Strau{\ss},
  Vieites \& Vences}]{Crottini2012}
Crottini, A., Madsen, O., Poux, C., Strau{\ss}, A., Vieites, D.R. \& Vences, M.
  (2012) {Vertebrate time-tree elucidates the biogeographic pattern of a major
  biotic change around the K--T boundary in Madagascar}.
\newblock \emph{Proceedings of the National Academy of Sciences}, \textbf{109},
  5358--5363.

\bibitem[{Eklund \emph{et~al.}(2016)Eklund, Blanchet, Nyman, Rocha, Virtanen \&
  Cabeza}]{Eklund2016}
Eklund, J., Blanchet, F.G., Nyman, J., Rocha, R., Virtanen, T. \& Cabeza, M.
  (2016) {Contrasting spatial and temporal trends of protected area
  effectiveness in mitigating deforestation in Madagascar}.
\newblock \emph{Biological Conservation}, \textbf{203}, 290 -- 297.

\bibitem[{Freudenberger(2010)}]{Freudenberger2010}
Freudenberger, K. (2010) {Paradise Lost? Lessons from 25 years of USAID
  environment programs in Madagascar}.
\newblock \emph{International Resources Group, Washington DC}.

\bibitem[{Gibson \emph{et~al.}(2013)Gibson, Lynam, Bradshaw, He, Bickford,
  Woodruff, Bumrungsri \& Laurance}]{Gibson2013}
Gibson, L., Lynam, A.J., Bradshaw, C.J., He, F., Bickford, D.P., Woodruff,
  D.S., Bumrungsri, S. \& Laurance, W.F. (2013) Near-complete extinction of
  native small mammal fauna 25 years after forest fragmentation.
\newblock \emph{Science}, \textbf{341}, 1508--1510.

\bibitem[{Goodman \& Benstead(2005)}]{Goodman2005}
Goodman, S.M. \& Benstead, J.P. (2005) {Updated estimates of biotic diversity
  and endemism for Madagascar}.
\newblock \emph{Oryx}, \textbf{39}, 73--77.

\bibitem[{Gorenflo \emph{et~al.}(2011)Gorenflo, Corson, Chomitz, Harper,
  Honzák \& {\"O}zler}]{Gorenflo2011}
Gorenflo, L.J., Corson, C., Chomitz, K.M., Harper, G., Honzák, M. \&
  {\"O}zler, B. (2011) \emph{{Exploring the Association Between People and
  Deforestation in Madagascar}}, vol. 1650.
\newblock Springer Berlin Heidelberg.

\bibitem[{Grinand \emph{et~al.}(2017)Grinand, Le~Maire, Vieilledent,
  Razakamanarivo, Razafimbelo \& Bernoux}]{Grinand2017}
Grinand, C., Le~Maire, G., Vieilledent, G., Razakamanarivo, H., Razafimbelo, T.
  \& Bernoux, M. (2017) {Estimating temporal changes in soil carbon stocks at
  ecoregional scale in Madagascar using remote-sensing}.
\newblock \emph{International Journal of Applied Earth Observation and
  Geoinformation}, \textbf{54}, 1--14.

\bibitem[{Grinand \emph{et~al.}(2013)Grinand, Rakotomalala, Gond, Vaudry,
  Bernoux \& Vieilledent}]{Grinand2013}
Grinand, C., Rakotomalala, F., Gond, V., Vaudry, R., Bernoux, M. \&
  Vieilledent, G. (2013) {Estimating deforestation in tropical humid and dry
  forests in Madagascar from 2000 to 2010 using multi-date Landsat satellite
  images and the Random Forests classifier}.
\newblock \emph{Remote Sensing of Environment}, \textbf{139}, 68--80.

\bibitem[{Grouzis \emph{et~al.}(2001)Grouzis, Razanaka, Le~Floc’h \&
  Leprun}]{Grouzis2001}
Grouzis, M., Razanaka, S., Le~Floc’h, E. \& Leprun, J.C. (2001)
  {{\'E}volution de la v{\'e}g{\'e}tation et de quelques param{\`e}tres
  {\'e}daphiques au cours de la phase post-culturale dans la r{\'e}gion
  d’Analabo}.
\newblock \emph{Soci{\'e}t{\'e}s paysannes, transitions agraires et dynamiques
  {\'e}cologiques dans le Sud-Ouest de Madagascar, Antananarivo, IRD/CNRE}, pp.
  327--337.

\bibitem[{Hansen \emph{et~al.}(2013)Hansen, Potapov, Moore, Hancher,
  Turubanova, Tyukavina, Thau, Stehman, Goetz, Loveland, Kommareddy, Egorov,
  Chini, Justice \& Townshend}]{Hansen2013}
Hansen, M.C., Potapov, P.V., Moore, R., Hancher, M., Turubanova, S.A.,
  Tyukavina, A., Thau, D., Stehman, S.V., Goetz, S.J., Loveland, T.R.,
  Kommareddy, A., Egorov, A., Chini, L., Justice, C.O. \& Townshend, J.R.G.
  (2013) {High-Resolution Global Maps of 21st-Century Forest Cover Change}.
\newblock \emph{Science}, \textbf{342}, 850--853.

\bibitem[{Harper \emph{et~al.}(2007)Harper, Steininger, Tucker, Juhn \&
  Hawkins}]{Harper2007}
Harper, G.J., Steininger, M.K., Tucker, C.J., Juhn, D. \& Hawkins, F. (2007)
  {Fifty years of deforestation and forest fragmentation in Madagascar}.
\newblock \emph{Environmental Conservation}, \textbf{34}, 325--333.

\bibitem[{Humbert(1927)}]{Humbert1927}
Humbert, H. (1927) {La destruction d'une flore insulaire par le feu. Principaux
  aspects de la végétation à Madagascar}.
\newblock \emph{Mémoires de l'Académie Malgache}, \textbf{5}, 1--80.

\bibitem[{Keenan \emph{et~al.}(2015)Keenan, Reams, Achard, de~Freitas, Grainger
  \& Lindquist}]{Keenan2015}
Keenan, R.J., Reams, G.A., Achard, F., de~Freitas, J.V., Grainger, A. \&
  Lindquist, E. (2015) {Dynamics of global forest area: Results from the FAO
  Global Forest Resources Assessment 2015}.
\newblock \emph{Forest Ecology and Management}, \textbf{352}, 9 -- 20.

\bibitem[{Kim \emph{et~al.}(2014)Kim, Sexton, Noojipady, Huang, Anand, Channan,
  Feng \& Townshend}]{Kim2014}
Kim, D.H., Sexton, J.O., Noojipady, P., Huang, C., Anand, A., Channan, S.,
  Feng, M. \& Townshend, J.R. (2014) {Global, Landsat-based forest-cover change
  from 1990 to 2000}.
\newblock \emph{Remote Sensing of Environment}, \textbf{155}, 178--193.

\bibitem[{Klein(2002)}]{Klein2002}
Klein, J. (2002) {Deforestation in the Madagascar Highlands -- Established
  `truth' and scientific uncertainty}.
\newblock \emph{GeoJournal}, \textbf{56}, 191--199.

\bibitem[{Kull(2000)}]{Kull2000}
Kull, C.A. (2000) {Deforestation, erosion, and fire: degradation myths in the
  environmental history of Madagascar}.
\newblock \emph{Environment and History}, \textbf{6}, 423--450.

\bibitem[{Perrier~de La~B{\^a}thie(1921)}]{Perrier1921}
Perrier~de La~B{\^a}thie, H. (1921) \emph{{La v{\'e}g{\'e}tation malgache}},
  vol.~23.
\newblock Mus{\'e}e Colonial.

\bibitem[{{MEFT, USAID, and CI}(2009)}]{MEFT2009}
{MEFT, USAID, and CI} (2009) \emph{{Evolution de la couverture de forêts
  naturelles à Madagascar, 1990-2000-2005}}.

\bibitem[{{Ministère de l'Environnement}(1996)}]{IEFN1996}
{Ministère de l'Environnement} (1996) {IEFN: Inventaire Ecologique Forestier
  National}.
\newblock {Ministère de l'Environnement de Madagascar, Direction des Eaux et
  Forêts, DFS Deutsch Forest Service GmbH, Entreprise d'études de
  développement rural ``Mamokatra'', FTM}.

\bibitem[{Minten \emph{et~al.}(2013)Minten, Sander \& Stifel}]{Minten2013}
Minten, B., Sander, K. \& Stifel, D. (2013) {Forest management and economic
  rents: Evidence from the charcoal trade in Madagascar}.
\newblock \emph{Energy for Sustainable Development}, \textbf{17}, 106 -- 115.

\bibitem[{Murcia(1995)}]{Murcia1995}
Murcia, C. (1995) Edge effects in fragmented forests: implications for
  conservation.
\newblock \emph{Trends in Ecology \& Evolution}, \textbf{10}, 58 -- 62.

\bibitem[{Neteler \& Mitasova(2008)}]{Neteler2008}
Neteler, M. \& Mitasova, H. (2008) \emph{{Open source GIS: a GRASS GIS
  approach}}.
\newblock Springer.

\bibitem[{Olander \emph{et~al.}(2008)Olander, Gibbs, Steininger, Swenson \&
  Murray}]{Olander2008}
Olander, L.P., Gibbs, H.K., Steininger, M., Swenson, J.J. \& Murray, B.C.
  (2008) {Reference scenarios for deforestation and forest degradation in
  support of REDD: a review of data and methods}.
\newblock \emph{Environmental Research Letters}, \textbf{3}, 025011.

\bibitem[{{ONE, DGF, FTM, MNP, and CI}(2013)}]{ONE2013}
{ONE, DGF, FTM, MNP, and CI} (2013) \emph{{Evolution de la couverture de
  forêts naturelles à Madagascar 2005-2010}}.

\bibitem[{{ONE, DGF, MNP, WCS, and Etc Terra}(2015)}]{ONE2015}
{ONE, DGF, MNP, WCS, and Etc Terra} (2015) \emph{{Changement de la couverture
  de forêts naturelles à Madagascar, 2005-2010-2013}}.

\bibitem[{Pearson \& Raxworthy(2009)}]{Pearson2009}
Pearson, R.G. \& Raxworthy, C.J. (2009) {The evolution of local endemism in
  Madagascar: watershed versus climatic gradient hypotheses evaluated by null
  biogeographic models}.
\newblock \emph{Evolution}, \textbf{63}, 959--967.

\bibitem[{Pekel \emph{et~al.}(2016)Pekel, Cottam, Gorelick \&
  Belward}]{Pekel2016}
Pekel, J.F., Cottam, A., Gorelick, N. \& Belward, A.S. (2016) {High-resolution
  mapping of global surface water and its long-term changes}.
\newblock \emph{Nature}, \textbf{540}, 418--422.

\bibitem[{Ploch \& Cook(2012)}]{Ploch2012}
Ploch, L. \& Cook, N. (2012) {Madagascar's Political Crisis}.

\bibitem[{Potapov \emph{et~al.}(2017)Potapov, Hansen, Laestadius, Turubanova,
  Yaroshenko, Thies, Smith, Zhuravleva, Komarova, Minnemeyer \&
  Esipova}]{Potapov2017}
Potapov, P., Hansen, M.C., Laestadius, L., Turubanova, S., Yaroshenko, A.,
  Thies, C., Smith, W., Zhuravleva, I., Komarova, A., Minnemeyer, S. \&
  Esipova, E. (2017) {The last frontiers of wilderness: Tracking loss of intact
  forest landscapes from 2000 to 2013}.
\newblock \emph{Science Advances}, \textbf{3}.

\bibitem[{Puyravaud(2003)}]{Puyravaud2003}
Puyravaud, J.P. (2003) Standardizing the calculation of the annual rate of
  deforestation.
\newblock \emph{Forest Ecology and Management}, \textbf{177}, 593--596.

\bibitem[{Rakotomala \emph{et~al.}(2015)Rakotomala, Rabenandrasana,
  Andriambahiny, Rajaonson, Andriamalala, Burren, Rakotoarijaona, Parany,
  Vaudry, Rakotoniaina, Ranaivosoa, Rahagalala, Randrianary \&
  Grinand}]{Rakotomalala2015}
Rakotomala, F., Rabenandrasana, J., Andriambahiny, J., Rajaonson, R.,
  Andriamalala, F., Burren, C., Rakotoarijaona, J., Parany, B., Vaudry, R.,
  Rakotoniaina, S., Ranaivosoa, R., Rahagalala, P., Randrianary, T. \& Grinand,
  C. (2015) {Estimation de la d{\'e}forestation des for{\^e}ts humides {\`a}
  Madagascar entre 2005, 2010 et 2013: combinaison multi-date d'images LANDSAT,
  utilisation de l'algorithme Random Forest et proc{\'e}dure de validation}.
\newblock \emph{Revue Fran{\c{c}}aise de Photogramm{\'e}trie et de
  T{\'e}l{\'e}d{\'e}tection}, pp. 11--23.

\bibitem[{Riitters \emph{et~al.}(2000)Riitters, Wickham, O'Neill, Jones \&
  Smith}]{Riitters2000}
Riitters, K., Wickham, J., O'Neill, R., Jones, B. \& Smith, E. (2000)
  Global-scale patterns of forest fragmentation.
\newblock \emph{Conservation Ecology}, \textbf{4}, 3.

\bibitem[{Saatchi \emph{et~al.}(2011)Saatchi, Harris, Brown, Lefsky, Mitchard,
  Salas, Zutta, Buermann, Lewis, Hagen, Petrova, White, Silman \&
  Morel}]{Saatchi2011}
Saatchi, S.S., Harris, N.L., Brown, S., Lefsky, M., Mitchard, E.T.A., Salas,
  W., Zutta, B.R., Buermann, W., Lewis, S.L., Hagen, S., Petrova, S., White,
  L., Silman, M. \& Morel, A. (2011) Benchmark map of forest carbon stocks in
  tropical regions across three continents.
\newblock \emph{Proceedings of the National Academy of Sciences}, \textbf{108},
  9899--9904.

\bibitem[{Saunders \emph{et~al.}(1991)Saunders, Hobbs \&
  Margules}]{Saunders1991}
Saunders, D.A., Hobbs, R.J. \& Margules, C.R. (1991) {Biological Consequences
  of Ecosystem Fragmentation: A Review}.
\newblock \emph{Conservation Biology}, \textbf{5}, 18--32.

\bibitem[{Scales(2011)}]{Scales2011}
Scales, I.R. (2011) {Farming at the Forest Frontier: Land Use and Landscape
  Change in Western Madagascar, 1896-2005}.
  \newblock \emph{Environment and History}, \textbf{17}, 499--524.

\bibitem[{Smith \emph{et~al.}(2003)Smith, Muir, Walpole, Balmford \&
  Leader-Williams}]{Smith2003}
Smith, R.J., Muir, R.D.J., Walpole, M.J., Balmford, A. \& Leader-Williams, N.
  (2003) Governance and the loss of biodiversity.
\newblock \emph{Nature}, \textbf{426}, 67--70.

\bibitem[{Tidd \emph{et~al.}(2001)Tidd, Pinder \& Ferguson}]{Tidd2001}
Tidd, S.T., Pinder, J. \& Ferguson, G.W. (2001) {Deforestation and habitat loss
  for the Malagasy flat-tailed tortoise from 1963 through 1993}.
\newblock \emph{Chelonian Conservation and Biology}, \textbf{4}, 59--65.

\bibitem[{Tropek \emph{et~al.}(2014)Tropek, Sedl{\'a}{\v c}ek, Beck, Keil,
  Musilov{\'a}, {\v S}{\'\i}mov{\'a} \& Storch}]{Tropek2014}
Tropek, R., Sedl{\'a}{\v c}ek, O., Beck, J., Keil, P., Musilov{\'a}, Z., {\v
  S}{\'\i}mov{\'a}, I. \& Storch, D. (2014) {Comment on "High-resolution global
  maps of 21st-century forest cover change"}.
\newblock \emph{Science}, \textbf{344}, 981--981.

\bibitem[{Tyukavina \emph{et~al.}(2017)Tyukavina, Hansen, Potapov, Stehman,
  Smith-Rodriguez, Okpa \& Aguilar}]{Tyukavina2017}
Tyukavina, A., Hansen, M.C., Potapov, P.V., Stehman, S.V., Smith-Rodriguez, K.,
  Okpa, C. \& Aguilar, R. (2017) {Types and rates of forest disturbance in
  Brazilian Legal Amazon, 2000{\textendash}2013}.
\newblock \emph{Science Advances}, \textbf{3}.

\bibitem[{{United Nations}(2015)}]{UN2015}
{United Nations} (2015) \emph{{World Population Prospects: The 2015 Revision,
  Key Findings and Advance Tables. Working Paper No. ESA/P/WP.241.}}

\bibitem[{Verhegghen \emph{et~al.}(2016)Verhegghen, Eva, Desclée \&
  Achard}]{Verhegghen2016}
Verhegghen, A., Eva, H., Desclée, B. \& Achard, F. (2016) {Review and
  Combination of Recent Remote Sensing Based Products for Forest Cover Change
  Assessments in Cameroon}.
\newblock \emph{International Forestry Review}, \textbf{18}, 14--25.

\bibitem[{Vieilledent \emph{et~al.}(2016)Vieilledent, Gardi, Grinand, Burren,
  Andriamanjato, Camara, Gardner, Glass, Rasolohery, Rakoto~Ratsimba, Gond \&
  Rakotoarijaona}]{Vieilledent2016}
Vieilledent, G., Gardi, O., Grinand, C., Burren, C., Andriamanjato, M., Camara,
  C., Gardner, C.J., Glass, L., Rasolohery, A., Rakoto~Ratsimba, H., Gond, V.
  \& Rakotoarijaona, J.R. (2016) {Bioclimatic envelope models predict a
  decrease in tropical forest carbon stocks with climate change in Madagascar}.
\newblock \emph{Journal of Ecology}, \textbf{104}, 703--715.

\bibitem[{Vieilledent \emph{et~al.}(2013)Vieilledent, Grinand \&
  Vaudry}]{Vieilledent2013}
Vieilledent, G., Grinand, C. \& Vaudry, R. (2013) {Forecasting deforestation
  and carbon emissions in tropical developing countries facing demographic
  expansion: a case study in Madagascar}.
\newblock \emph{Ecology and Evolution}, \textbf{3}, 1702--1716.

\bibitem[{Virah-Sawmy(2009)}]{Virah-Sawmy2009}
Virah-Sawmy, M. (2009) {Ecosystem management in Madagascar during global
  change}.
\newblock \emph{Conservation Letters}, \textbf{2}, 163--170.

\bibitem[{Vorontsova \emph{et~al.}(2016)Vorontsova, Besnard, Forest, Malakasi,
  Moat, Clayton, Ficinski, Savva, Nanjarisoa, Razanatsoa, Randriatsara, Kimeu,
  Luke, Kayombo \& Linder}]{Vorontsova2017}
Vorontsova, M.S., Besnard, G., Forest, F., Malakasi, P., Moat, J., Clayton,
  W.D., Ficinski, P., Savva, G.M., Nanjarisoa, O.P., Razanatsoa, J.,
  Randriatsara, F.O., Kimeu, J.M., Luke, W.R.Q., Kayombo, C. \& Linder, H.P.
  (2016) {Madagascar's grasses and grasslands: anthropogenic or natural?}
\newblock \emph{Proceedings of the Royal Society of London B: Biological
  Sciences}, \textbf{283}.

\end{thebibliography}


\newpage

\section{Tables}
\label{tables}

\nopagebreak

\vfill
\begin{table}[!h]
  \begin{longtable}[]{@{}rrrrr@{}}
    \toprule
    Year & Forest (ha) & Unmapped (ha) & Annual defor. (ha/yr) & Rate
    (\%/yr)\tabularnewline
    \midrule
    \endhead
    1953 & 15968176 & 0 & - & -\tabularnewline
    1973 & 14242592 & 3316531 & 86279 & 0.57\tabularnewline
    1990 & 10762442 & 0 & 204715 & 1.63\tabularnewline
    2000 & 9879031 & 0 & 88341 & 0.85\tabularnewline
    2005 & 9667553 & 0 & 42296 & 0.43\tabularnewline
    2010 & 9319851 & 0 & 69540 & 0.73\tabularnewline
    2014 & 8925246 & 0 & 98651 & 1.08\tabularnewline
    \bottomrule
  \end{longtable}
  \addtocounter{table}{-1}

  \caption{\textbf{Evolution of the forest cover and deforestation
      rates from 1953 to 2014 in Madagascar}. Forest map for the year
    1973 has 3.3 Mha of unclassified areas due to the presence of
    clouds on satellite images. As a consequence, deforestation rates
    for the periods 1953-1973 and 1973-1990 are indicative. The two
    last columns indicate the annual deforested areas and annual
    deforestation rates on the previous time-period (e.g.~1953-1973
    for year 1973, 1973-1990 for year 1990, etc.).}

  \label{tab:forest_cover}
\end{table}
\vfill

\newpage

\vspace*{\stretch{1}}
\begin{table}[!h]
  {\footnotesize
  \begin{longtable}[]{@{}llrrrrrrrr@{}}
    \toprule
    Forest type & Source & 1953 & 1973 & 1990 & 2000 & 2005 & 2010 & 2013 &
    2014\tabularnewline
    \midrule
    \endhead
    Total & Harper2007 & 15995900 & 14173100 & 10605700 & 8982100 & - & - &
    - & -\tabularnewline
    & MEFT2009 & - & - & 10650142 & 9678402 & 9413218 & - & - &
    -\tabularnewline
    & ONE2015 & - & - & - & - & 9451350 & 8977337 & 8485509 &
    -\tabularnewline
    & this study & 15968176 & 14242592 & 10762494 & 9879031 & 9667553 &
    9319851 & 9051029 & 8925246\tabularnewline
    Moist & Harper2007 & 8765600 & 6876000 & 5234300 & 4166800 & - & - & - &
    -\tabularnewline
    & MEFT2009 & - & - & 5270599 & 4787771 & 4700430 & - & - &
    -\tabularnewline
    & ONE2015 & - & - & - & - & 4555788 & 4457184 & 4345093 &
    -\tabularnewline
    & this study & 8578299 & 6989942 & 5270169 & 4872016 & 4767876 & 4633104
    & 4470194 & 4409842\tabularnewline
    Dry & Harper2007 & 4252100 & 4027700 & 2711800 & 2457000 & - & - & - &
    -\tabularnewline
    & MEFT2009 & - & - & 3320582 & 3084976 & 3027505 & - & - &
    -\tabularnewline
    & ONE2015 & - & - & - & - & 3223028 & 2970192 & 2678640 &
    -\tabularnewline
    & this study & 4761551 & 4434871 & 3224917 & 2940970 & 2880819 & 2734639
    & 2642253 & 2595621\tabularnewline
    Spiny & Harper2007 & 2978200 & 3029800 & 2420000 & 2132200 & - & - & - &
    -\tabularnewline
    & MEFT2009 & - & - & 2123630 & 1871735 & 1756884 & - & - &
    -\tabularnewline
    & ONE2015 & - & - & - & - & 1681527 & 1558533 & 1466765 &
    -\tabularnewline
    & this study & 2462830 & 2582880 & 2054724 & 1857628 & 1810704 & 1744427
    & 1731308 & 1712731\tabularnewline
    Mangroves & Harper2007 & - & - & 239600 & 226100 & - & - & - &
    -\tabularnewline
    & MEFT2009 & - & - & - & - & - & - & - & -\tabularnewline
    & ONE2015 & - & - & - & - & 173564 & 171220 & 169877 & -\tabularnewline
    & this study & 143412 & 199853 & 181226 & 177708 & 177492 & 177149 &
    176890 & 176718\tabularnewline
    \bottomrule
  \end{longtable}}
  \addtocounter{table}{-1}

  \caption{\textbf{Comparing our estimates of forest-cover (in ha) for
      Madagascar with previous studies on the period 1953-2014}. We
    compared our estimates of forest-cover with the estimates from
    three previous studies \citep{Harper2007, MEFT2009, ONE2015}. We
    obtained a Pearson's correlation coefficient of 0.99 between our
    forest-cover estimates and forest-cover estimates from previous
    studies. The increase in mangrove and spiny forest covers from
    \emph{c.} 1953 to \emph{c.} 1973 in \citet{Harper2007} and our
    study is most probably due to differences in forest definition and
    mapping methods between the 1953 aerial-photography derived map
    and the 1973 Landsat image derived map.}

  \label{tab:comp_forest}
\end{table}
\vspace*{\stretch{1}}

\newpage

\vspace*{\stretch{1}}
\begin{table}[!h]
  {\footnotesize
  \begin{longtable}[]{@{}llrrrrrr@{}}
    \toprule
    Forest type & Source & 1953-1973 & 1973-1990 & 1990-2000 & 2000-2005 &
    2005-2010 & 2010-2013\tabularnewline
    \midrule
    \endhead
    Total & Harper2007 & 91140 (0.30) & 200206 (1.70) & 80740 (0.90) & - & -
    & -\tabularnewline
    & MEFT2009 & - & - & 97174 (0.83) & 53037 (0.53) & - & -\tabularnewline
    & ONE2015 & - & - & - & - & 94803 (1.18) & 163943 (1.50)\tabularnewline
    & this study & 86279 (0.57) & 204712 (1.63) & 88346 (0.85) & 42296
    (0.43) & 69540 (0.73) & 89607 (0.97)\tabularnewline
    Moist & Harper2007 & 94480 (0.60) & 87188 (1.70) & 32200 (0.80) & - & -
    & -\tabularnewline
    & MEFT2009 & - & - & 48283 (0.79) & 17468 (0.35) & - & -\tabularnewline
    & ONE2015 & - & - & - & - & 19721 (0.50) & 37364 (0.94)\tabularnewline
    & this study & 79418 (1.02) & 101163 (1.65) & 39815 (0.78) & 20828
    (0.43) & 26954 (0.57) & 54303 (1.19)\tabularnewline
    Dry & Harper2007 & 11220 (0.20) & 77153 (1.90) & 19820 (0.70) & - & - &
    -\tabularnewline
    & MEFT2009 & - & - & 23561 (0.67) & 11494 (0.40) & - & -\tabularnewline
    & ONE2015 & - & - & - & - & 50567 (1.80) & 97184 (2.29)\tabularnewline
    & this study & 16334 (0.35) & 71174 (1.86) & 28395 (0.92) & 12030 (0.41)
    & 29236 (1.04) & 30795 (1.14)\tabularnewline
    Spiny & Harper2007 & -2580 (-0.10) & 35865 (1.20) & 28170 (1.20) & - & -
    & -\tabularnewline
    & MEFT2009 & - & - & 25190 (1.19) & 22970 (1.23) & - & -\tabularnewline
    & ONE2015 & - & - & - & - & 24599 (1.69) & 30589 (1.66)\tabularnewline
    & this study & -6002 (-0.24) & 31068 (1.34) & 19710 (1.00) & 9385 (0.51)
    & 13255 (0.74) & 4373 (0.25)\tabularnewline
    Mangroves & Harper2007 & - & - & 550 (0.20) & - & - & -\tabularnewline
    & MEFT2009 & - & - & - & - & - & -\tabularnewline
    & ONE2015 & - & - & - & - & 469 (0.32) & 448 (0.20)\tabularnewline
    & this study & -2822 (-1.67) & 1096 (0.57) & 352 (0.20) & 43 (0.02) & 69
    (0.04) & 86 (0.05)\tabularnewline
    \bottomrule
  \end{longtable}}
  \addtocounter{table}{-1}

  \caption{\textbf{Comparing our estimates of annual deforestation
      rates for Madagascar with previous studies on the period
      1953-2014}. Annual deforestation areas (in ha/yr) and annual
    deforestation rates (second number in parenthesis, in \%/yr) are
    provided. For deforestation rates in \%/yr, exact same numbers as
    in scientific articles and reports from previous studies
    \citep{Harper2007, MEFT2009, ONE2015} have been reported. The way
    annual deforestation rates in \%/yr have been computed in these
    previous studies can slightly differ from one study to another, but
    estimates always correct for the potential presences of clouds on
    satellite images and unclassified areas on forest maps. Annual
    deforested areas in ha/yr have been recomputed from forest-cover
    estimates in Tab.~\ref{tab:comp_forest} (except for
    \citet{Harper2007} for the periods 1973-1990 and 1990-2000 for
    which annual deforested areas in ha/yr were derived from numbers
    reported in the original publication, see methods) and do not
    correct for the potential presence of clouds.}

  \label{tab:comp_defor}
\end{table}
\vspace*{\stretch{1}}

\newpage

\vspace*{\stretch{1}}
\begin{table}[!h]

  {\small
  \begin{longtable}[]{@{}rrrrrrrr@{}}
    \toprule
    Year & Forest (ha) & patch (\%) & transitional (\%) & edge (\%) &
    perforated (\%) & interior (\%) & NA (\%)\tabularnewline
    \midrule
    \endhead
    1953 & 15962870 & 0.01 & 1.12 & 4.46 & 0.58 & 93.83 &
    0.00\tabularnewline
    1973 & 14228217 & 2.21 & 7.25 & 19.81 & 2.86 & 67.87 &
    0.01\tabularnewline
    1990 & 10749572 & 3.00 & 8.17 & 21.28 & 3.81 & 63.73 &
    0.01\tabularnewline
    2000 & 9866145 & 3.09 & 8.37 & 22.13 & 3.92 & 62.49 &
    0.01\tabularnewline
    2005 & 9659861 & 3.51 & 8.88 & 22.56 & 6.44 & 58.59 &
    0.02\tabularnewline
    2010 & 9306528 & 4.28 & 9.72 & 22.94 & 8.52 & 54.52 &
    0.02\tabularnewline
    2014 & 8911481 & 5.18 & 10.72 & 23.25 & 10.58 & 50.24 &
    0.03\tabularnewline
    \bottomrule
  \end{longtable}}
  \addtocounter{table}{-1}

  \caption{\textbf{Evolution of the forest fragmentation from 1953 to 2014
      in Madagascar}. Six categories of fragmentation were identified from the
    amount of forest and its occurrence as adjacent forest pixels:
    ``interior'', ``perforated'', ``edge'', ``transitional'', ``patch'', and
    ``undetermined'' \citep{Riitters2000}. We used a moving window of 7x7
    pixels (4.4~ha). Using this window size, forest edge had a width of
    about 90~m. The ``interior'' category can be interpreted as the most
    intact forest. The ``patch'' and ``transitional'' categories correspond
    to isolated small forest patches.}

  \label{tab:frag}
\end{table}
\vspace*{\stretch{1}}

\newpage

\section{Figures}
\label{figures}

\nopagebreak

\vfill
\begin{figure}[h!]
  \centering
    
    \includegraphics[width=11cm]{outputs/ecoregion.png}
    
    \caption{\textbf{Ecoregions and forest types in Madagascar.}
      Madagascar can be divided into four climatic ecoregions with four forest
      types: the moist forest in the East (green), the dry forest in the West
      (orange), the spiny forest in the South (red), and the mangroves on the
      West coast (blue). Ecoregions were defined following climatic
      \citep{Cornet1974} and vegetation \citep{IEFN1996} criteria. The dark
      grey areas represent the remaining natural forest cover for the year
      2014.}

    \label{fig:ecoregion}
    
\end{figure}
\vfill

\newpage

\vspace*{\stretch{1}}
\begin{figure}[h!]
  \centering
  
  \includegraphics[width=\textwidth]{outputs/fig_fcc_highres.png}
  
  \caption{\textbf{Forest-cover change on six decades from 1953
      to 2014 in Madagascar.} Forest cover changes from \emph{c.} 1973 to 2014
    are shown in the main figure, and forest cover in \emph{c.} 1953 is
    shown in the bottom-right inset. Two zooms in the western dry (left
    part) and eastern moist (right part) ecoregions present more detailed
    views of (from top to bottom): forest-cover in 1950s, forest-cover
    change from \emph{c.} 1973 to 2014, forest fragmentation in 2014 and
    distance to forest edge in 2014. Data on water bodies (blue) and water
    seasonality (light blue for seasonal water to dark blue for permanent
    water) has been extracted from \citet{Pekel2016}.}

  \label{fig:fcc}

\end{figure}
\vspace*{\stretch{1}}

\newpage

\vspace*{\stretch{1}}
\begin{figure}[h!]
  \centering
  
  \includegraphics[width=13cm]{outputs/dist.png}
  
  \caption{\textbf{Evolution of the distance to forest edge from
      1953 to 2014 in Madagascar.} Black dots represent the mean distance to
    forest edge for each year. Vertical dashed segments represent the 90\%
    quantiles (5\% and 95\%) of the distance to forest edge. Horizontal
    dashed grey line indicates a distance to forest edge of 100~m.
    Percentages indicate the percentage of forest at a distance to forest
    edge lower than 100~m for each year.}

  \label{fig:dist_edge}

\end{figure}
\vspace*{\stretch{1}}

\end{document}
